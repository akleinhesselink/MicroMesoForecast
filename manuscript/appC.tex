\documentclass[12pt,]{article}
\usepackage[T1]{fontenc}
\usepackage{lmodern}
\usepackage{amssymb,amsmath}
\usepackage{ifxetex,ifluatex}
\usepackage{fixltx2e} % provides \textsubscript
% use upquote if available, for straight quotes in verbatim environments
\IfFileExists{upquote.sty}{\usepackage{upquote}}{}
\ifnum 0\ifxetex 1\fi\ifluatex 1\fi=0 % if pdftex
  \usepackage[utf8]{inputenc}
\else % if luatex or xelatex
  \ifxetex
    \usepackage{mathspec}
    \usepackage{xltxtra,xunicode}
  \else
    \usepackage{fontspec}
  \fi
  \defaultfontfeatures{Mapping=tex-text,Scale=MatchLowercase}
  \newcommand{\euro}{€}
\fi
% use microtype if available
\IfFileExists{microtype.sty}{\usepackage{microtype}}{}
\usepackage[margin=1in]{geometry}
\usepackage{graphicx}
% Redefine \includegraphics so that, unless explicit options are
% given, the image width will not exceed the width of the page.
% Images get their normal width if they fit onto the page, but
% are scaled down if they would overflow the margins.
\makeatletter
\def\ScaleIfNeeded{%
  \ifdim\Gin@nat@width>\linewidth
    \linewidth
  \else
    \Gin@nat@width
  \fi
}
\makeatother
\let\Oldincludegraphics\includegraphics
{%
 \catcode`\@=11\relax%
 \gdef\includegraphics{\@ifnextchar[{\Oldincludegraphics}{\Oldincludegraphics[width=\ScaleIfNeeded]}}%
}%
\ifxetex
  \usepackage[setpagesize=false, % page size defined by xetex
              unicode=false, % unicode breaks when used with xetex
              xetex]{hyperref}
\else
  \usepackage[unicode=true]{hyperref}
\fi
\hypersetup{breaklinks=true,
            bookmarks=true,
            pdfauthor={},
            pdftitle={},
            colorlinks=true,
            citecolor=blue,
            urlcolor=blue,
            linkcolor=magenta,
            pdfborder={0 0 0}}
\urlstyle{same}  % don't use monospace font for urls
\setlength{\parindent}{0pt}
\setlength{\parskip}{6pt plus 2pt minus 1pt}
\setlength{\emergencystretch}{3em}  % prevent overfull lines
\setcounter{secnumdepth}{0}

\author{}
\date{}
\usepackage{lineno}
\linenumbers
\usepackage{setspace}

\begin{document}

\normalsize


\renewcommand\thefigure{C\arabic{figure}} 



\section{Appendix C: Effect of recruitment on IPM
uncertainty}\label{appendix-c-effect-of-recruitment-on-ipm-uncertainty}

In Fig. 4 of the main text where we show the uncertainty around our
models' forecasts to a 1\% change in climate covariates, the obvious
outlier is \emph{Pascopyrum smithii} (PASM). The uncertainty around PASM
forecasts from the integral projection model is orders of magnitude
larger than for any other species, and is even larger than the
uncertainty around the quad-based model forecasts. It turns out that the
estimated dispersion parameter for the negative binomial distribution is
much higher for PASM than for other species (Appendix B). This is
because PASM has higher recruitment rates than the other species and the
relationship between number of recruits and quadrat cover (see main
text) is weak. So, when we draw parameters from the MCMC chain for our
simulations, even larger values for the dispersion parameter are
possible. Thus, there can be very large ``boom'' years where recruitment
is high. When we allow all parameters to vary except those associated
with recruitment, uncertainty around PASM decreases to levels similar to
those associated with other species (Fig. C1).

\begin{figure}[htbp]
\centering
\includegraphics{appC/figure/appC-plot_it.pdf}
\caption{Mean (points) and 90\% quantiles (errorbars) for the
proportional difference between baseline simulations (using observed
climate) and the climate pertubation simulation on the x-axis. We
calculated proportional difference as log(perturbed climate cover) -
log(observed climate cover), where `perturbed' and `observed' refer to
the climate time series used to drive interannual variation in the
simulations. Model error and parameter uncertainty were propagated
through the simulation phase for all vital rates except recruitment.}
\end{figure}

\end{document}

\documentclass[12pt,]{article}
\usepackage[T1]{fontenc}
\usepackage{lmodern}
\usepackage{amssymb,amsmath}
\usepackage{ifxetex,ifluatex}
\usepackage{fixltx2e} % provides \textsubscript
% use upquote if available, for straight quotes in verbatim environments
\IfFileExists{upquote.sty}{\usepackage{upquote}}{}
\ifnum 0\ifxetex 1\fi\ifluatex 1\fi=0 % if pdftex
  \usepackage[utf8]{inputenc}
\else % if luatex or xelatex
  \ifxetex
    \usepackage{mathspec}
    \usepackage{xltxtra,xunicode}
  \else
    \usepackage{fontspec}
  \fi
  \defaultfontfeatures{Mapping=tex-text,Scale=MatchLowercase}
  \newcommand{\euro}{€}
\fi
% use microtype if available
\IfFileExists{microtype.sty}{\usepackage{microtype}}{}
\usepackage[margin=1in]{geometry}
\usepackage{color}
\usepackage{fancyvrb}
\newcommand{\VerbBar}{|}
\newcommand{\VERB}{\Verb[commandchars=\\\{\}]}
\DefineVerbatimEnvironment{Highlighting}{Verbatim}{commandchars=\\\{\}}
% Add ',fontsize=\small' for more characters per line
\usepackage{framed}
\definecolor{shadecolor}{RGB}{248,248,248}
\newenvironment{Shaded}{\begin{snugshade}}{\end{snugshade}}
\newcommand{\KeywordTok}[1]{\textcolor[rgb]{0.13,0.29,0.53}{\textbf{{#1}}}}
\newcommand{\DataTypeTok}[1]{\textcolor[rgb]{0.13,0.29,0.53}{{#1}}}
\newcommand{\DecValTok}[1]{\textcolor[rgb]{0.00,0.00,0.81}{{#1}}}
\newcommand{\BaseNTok}[1]{\textcolor[rgb]{0.00,0.00,0.81}{{#1}}}
\newcommand{\FloatTok}[1]{\textcolor[rgb]{0.00,0.00,0.81}{{#1}}}
\newcommand{\CharTok}[1]{\textcolor[rgb]{0.31,0.60,0.02}{{#1}}}
\newcommand{\StringTok}[1]{\textcolor[rgb]{0.31,0.60,0.02}{{#1}}}
\newcommand{\CommentTok}[1]{\textcolor[rgb]{0.56,0.35,0.01}{\textit{{#1}}}}
\newcommand{\OtherTok}[1]{\textcolor[rgb]{0.56,0.35,0.01}{{#1}}}
\newcommand{\AlertTok}[1]{\textcolor[rgb]{0.94,0.16,0.16}{{#1}}}
\newcommand{\FunctionTok}[1]{\textcolor[rgb]{0.00,0.00,0.00}{{#1}}}
\newcommand{\RegionMarkerTok}[1]{{#1}}
\newcommand{\ErrorTok}[1]{\textbf{{#1}}}
\newcommand{\NormalTok}[1]{{#1}}
\usepackage{longtable,booktabs}
\ifxetex
  \usepackage[setpagesize=false, % page size defined by xetex
              unicode=false, % unicode breaks when used with xetex
              xetex]{hyperref}
\else
  \usepackage[unicode=true]{hyperref}
\fi
\hypersetup{breaklinks=true,
            bookmarks=true,
            pdfauthor={},
            pdftitle={},
            colorlinks=true,
            citecolor=blue,
            urlcolor=blue,
            linkcolor=magenta,
            pdfborder={0 0 0}}
\urlstyle{same}  % don't use monospace font for urls
\setlength{\parindent}{0pt}
\setlength{\parskip}{6pt plus 2pt minus 1pt}
\setlength{\emergencystretch}{3em}  % prevent overfull lines
\setcounter{secnumdepth}{0}

\author{}
\date{}
\usepackage{lineno}
\linenumbers
\usepackage{setspace}

\begin{document}

\normalsize


\renewcommand\thefigure{A\arabic{figure}}  

\section{Appendix A: Detailed model
descriptions}\label{appendix-a-detailed-model-descriptions}

Here we describe the statistical models in complete detail, including
full description of the Bayesian models and Stan code for each model. We
write each model as three parts: (1) a data model (likelihood), (2) a
process model, and (3) a parameter model. We then combine these three
models to write the posterior and joint distributions as a single model
statement. Following the full model expression we provide the Stan code
that directly corresponds the statement of posterior and joint
distributions. In our expressions of the posterior and joint
distributions we use general probability notation where {[}.{]}
represents some unspecified distribution. We do this for clarity of
model presentation. All prior distributions are defined in the parameter
model sections. For clarity we avoid subscripting species (the \emph{j}
subscripts) after initial description of each model.

\subsection{Parameter to model notation
keys}\label{parameter-to-model-notation-keys}

\begin{longtable}[c]{@{}lll@{}}
\toprule\addlinespace
Parameter & Definition & Notation in statistical results tables
\\\addlinespace
\midrule\endhead
$\gamma_{t}$ & intercept for year \emph{t} & \texttt{a{[}t{]}} where
\texttt{t} is the numeric year
\\\addlinespace
$\tilde{\gamma}$ & global intercept & \texttt{a\_mu}
\\\addlinespace
$\phi_{Q}$ & random effect of quadrat group \emph{Q} &
\texttt{gint{[}Q{]}} where \texttt{Q} is the quadrat group
\\\addlinespace
$\beta_{t}$ & size effect for year \emph{t} & \texttt{b1{[}t{]}} where
\texttt{t} is the numeric year
\\\addlinespace
$\tilde{\beta}$ & global size effect & \texttt{b1\_mu}
\\\addlinespace
$\gamma$ & effect of crowding & \texttt{w{[}1{]}}
\\\addlinespace
$\nu$ & crowding $\times$ size effect & \texttt{w{[}2{]}}
\\\addlinespace
$\theta_{k}$ & effect of climate covariate \emph{k} & \texttt{b2{[}k{]}}
where \texttt{k} is the climate effect
\\\addlinespace
\emph{a} & first parameter for growth variance & \texttt{tau}
\\\addlinespace
\emph{b} & second parameter for growth variance & \texttt{tauSize}
\\\addlinespace
\bottomrule
\addlinespace
\caption{Definition of parameters and model notation for IPM survival
and growth models.}
\end{longtable}

\begin{longtable}[c]{@{}lll@{}}
\toprule\addlinespace
Parameter & Definition & Notation in statistical results tables
\\\addlinespace
\midrule\endhead
$\gamma_{t}$ & intercept for year \emph{t} & \texttt{a{[}t{]}} where
\texttt{t} is the numeric year
\\\addlinespace
$\tilde{\gamma}$ & global intercept & \texttt{a\_mu}
\\\addlinespace
$\phi_{Q}$ & random effect of quadrat group \emph{Q} &
\texttt{gint{[}Q{]}} where \texttt{Q} is the quadrat group
\\\addlinespace
$\gamma$ & effect of plot cover & \texttt{dd}
\\\addlinespace
$\theta_{k}$ & effect of climate covariate \emph{k} & \texttt{b2{[}k{]}}
where \texttt{k} is the climate effect
\\\addlinespace
\emph{p} & mixing fraction for effective plot cover & \texttt{u}
\\\addlinespace
$\zeta$ & size parameter for negative binimial likelihood &
\texttt{theta}
\\\addlinespace
\bottomrule
\addlinespace
\caption{Definition of parameters and model notation for IPM recruitment
model.}
\end{longtable}

\begin{longtable}[c]{@{}lll@{}}
\toprule\addlinespace
Parameter & Definition & Notation in statistical results tables
\\\addlinespace
\midrule\endhead
$\gamma_{t}$ & intercept for year \emph{t} & \texttt{a{[}t{]}} where
\texttt{t} is the numeric year
\\\addlinespace
$\tilde{\gamma}$ & global intercept & \texttt{a\_mu}
\\\addlinespace
$\phi_{Q}$ & random effect of quadrat group \emph{Q} &
\texttt{gint{[}Q{]}} where \texttt{Q} is the quadrat group
\\\addlinespace
$\beta_{t}$ & cover effect for year \emph{t} & \texttt{b1{[}t{]}} where
\texttt{t} is the numeric year
\\\addlinespace
$\tilde{\beta}$ & global cover effect & \texttt{b1\_mu}
\\\addlinespace
$\theta_{k}$ & effect of climate covariate \emph{k} & \texttt{b2{[}k{]}}
where \texttt{k} is the climate effect
\\\addlinespace
$\tau$ & model variance in log normal likelihood & \texttt{tau}
\\\addlinespace
\bottomrule
\addlinespace
\caption{Definition of parameters and model notation for quadrat based
model.}
\end{longtable}

\begin{longtable}[c]{@{}ll@{}}
\toprule\addlinespace
Integer ID (\emph{k}) & Climate covariate
\\\addlinespace
\midrule\endhead
1 & pptLag
\\\addlinespace
2 & ppt1
\\\addlinespace
3 & ppt2
\\\addlinespace
4 & TmeanSpr1
\\\addlinespace
5 & TmeanSpr2
\\\addlinespace
6 & ppt1 $\times$ TmeanSpr1
\\\addlinespace
7 & ppt2 $\times$ TmeanSpr2
\\\addlinespace
8 & pptLag $\times$ size
\\\addlinespace
9 & ppt1 $\times$ size
\\\addlinespace
10 & ppt2 $\times$ size
\\\addlinespace
11 & TmeanSpr1 $\times$ size
\\\addlinespace
12 & TmeanSpr2 $\times$ size
\\\addlinespace
\bottomrule
\addlinespace
\caption{Climate effect key.}
\end{longtable}

\subsection{Survival (IPM)}\label{survival-ipm}

We used logistic regression to model survival probability ($S$) of genet
$i$ from species $j$ in quadrat group $Q$ from time $t$ to $t+1$:

\begin{align}
\text{logit}(S_{ijQ,t}) &= \gamma^{S}_{j,t} + \phi^{S}_{jQ} + \beta^{S}_{j,t}x_{ij,t} + \omega^{S}_{j}w_{ij,t} + \nu^{S}_{j}w_{ij,t}x_{ij,t} + \theta^{S}_{jk}C_{k,t} \\
y^{S}_{ijQ,t} &\sim \text{Bernoulli}(S_{ijQ,t})
\end{align}

where $x_{ij,t}$ is the log of genet size, $\gamma^{S}_{j,t}$ is a
year-specific intercept, $\beta^{S}_{j,t}$ is the year-specific slope
parameter for size, $\phi^{S}_{jQ}$ is the random effect of quadrat
group location, and $\theta^{S}_{k}$ is the fixed parameter for the
effect of the $k$th climate covariate at time $t$ ($C_{k,t}$). Note that
the vector of climate covariates (\textbf{C}) includes climate variable
interactions and climate$\times$size interactions. We include
density-dependence by estimating the effect of crowding on the focal
individual by other individuals of the same species. $\omega$ is the
effect of crowding and $w_{t,Q}$ is the crowding experienced by the
focal individual at time $t$ in quadrat group $Q$. We include a
size$\times$crowding interaction effect ($\nu^{S}$).

\textbf{Data, process, and parameter models}

\begin{align}
&\mathtt{data} \qquad &\textbf{y} \sim \text{Bernoulli}(\boldsymbol{\mu}) \\
&\mathtt{process} \qquad &\text{logit}(\mu_{iQ,t}) = \gamma_{t} + \phi_{Q} + \beta_{t}x_{i,t} + \omega w_{i,t} + \nu w_{i,t}x_{i,t} + \theta_{k}C_{k,t} \\
&\mathtt{parameters} \qquad &\gamma_{t} \sim \text{Normal}(\tilde{\gamma}, \sigma_\gamma^2)\\
& \qquad & \phi_{Q} \sim \text{Normal}(0,\sigma_Q^2) \\
& \qquad & \beta_{t} \sim \text{Normal}(\tilde{\beta}, \sigma_\beta^2) \\
& \qquad & \omega \sim \text{Uniform}(-100, 100) \\
& \qquad & \nu \sim \text{Uniform}(-100, 100) \\
& \qquad & \boldsymbol{\theta} \sim \text{Uniform}(-10, 10) \\
& \qquad & \tilde{\gamma} \sim \text{Uniform}(-300, 300) \\
& \qquad & \tilde{\beta} \sim \text{Uniform}(-100, 100) \\
& \qquad & \sigma_Q^2 \sim \text{Cauchy}(0, 5) \\
& \qquad & \sigma_\gamma^2 \sim \text{Cauchy}(0, 5) \\
& \qquad & \sigma_\beta^2 \sim \text{Cauchy}(0, 5)
\end{align}

\textbf{Full expression of posterior and joint distributions}

\begin{align}
[\boldsymbol{\gamma}, \tilde{\gamma}, \boldsymbol{\phi}, \boldsymbol{\beta}, \tilde{\beta}, \omega, \boldsymbol{\theta}, \sigma_Q^2, \sigma_\beta^2] &\propto \\
&\prod_{t=1}^T \prod_{i=1}^n [y_{iQ,t} | \gamma_{t}, \phi_{Q}, \beta_{t}, \omega, \boldsymbol{\theta}] [\gamma_{t} | \tilde{\gamma}] [\beta_{t} | \tilde{\beta}, \sigma_\beta^2] \times \\
&\prod_{Q=1}^{Q_{tot}} [\phi_{Q} | \sigma_Q^2] \times \\
&[\boldsymbol{\theta}] [\omega] [\tilde{\gamma}] [\tilde{\beta}] [\sigma_Q^2] [\sigma_\beta^2]
\end{align}

\textbf{Stan code for model}

\begin{Shaded}
\begin{Highlighting}[]
\NormalTok{data\{}
  \NormalTok{int<lower=}\DecValTok{0}\NormalTok{>}\StringTok{ }\NormalTok{N; /}\ErrorTok{/}\StringTok{ }\NormalTok{observations}
  \NormalTok{int<lower=}\DecValTok{0}\NormalTok{>}\StringTok{ }\NormalTok{Yrs; /}\ErrorTok{/}\StringTok{ }\NormalTok{years}
  \NormalTok{int<lower=}\DecValTok{0}\NormalTok{>}\StringTok{ }\NormalTok{yid[N]; /}\ErrorTok{/}\StringTok{ }\NormalTok{year id}
  \NormalTok{int<lower=}\DecValTok{0}\NormalTok{>}\StringTok{ }\NormalTok{Covs; /}\ErrorTok{/}\StringTok{ }\NormalTok{climate covariates}
  \NormalTok{int<lower=}\DecValTok{0}\NormalTok{>}\StringTok{ }\NormalTok{G; /}\ErrorTok{/}\StringTok{ }\NormalTok{groups}
  \NormalTok{int<lower=}\DecValTok{0}\NormalTok{>}\StringTok{ }\NormalTok{gid[N]; /}\ErrorTok{/}\StringTok{ }\NormalTok{group id}
  \NormalTok{int<lower=}\DecValTok{0}\NormalTok{,upper=}\DecValTok{1}\NormalTok{>}\StringTok{ }\NormalTok{Y[N]; /}\ErrorTok{/}\StringTok{ }\NormalTok{observation vector}
  \NormalTok{matrix[N,Covs] C; /}\ErrorTok{/}\StringTok{ }\NormalTok{climate matrix}
  \NormalTok{vector[N] X; /}\ErrorTok{/}\StringTok{ }\NormalTok{size vector}
  \NormalTok{matrix[N,}\DecValTok{2}\NormalTok{] W; /}\ErrorTok{/}\StringTok{ }\NormalTok{crowding matrix}
\NormalTok{\}}
\NormalTok{parameters\{}
  \NormalTok{real a_mu;}
  \NormalTok{vector[Yrs] a;}
  \NormalTok{real b1_mu;}
  \NormalTok{vector[Yrs] b1;}
  \NormalTok{vector[Covs] b2;}
  \NormalTok{vector[}\DecValTok{2}\NormalTok{] w;}
  \NormalTok{vector[G] gint;}
  \NormalTok{real<lower=}\DecValTok{0}\NormalTok{>}\StringTok{ }\NormalTok{sig_a;}
  \NormalTok{real<lower=}\DecValTok{0}\NormalTok{>}\StringTok{ }\NormalTok{sig_b1;}
  \NormalTok{real<lower=}\DecValTok{0}\NormalTok{>}\StringTok{ }\NormalTok{sig_G;}
\NormalTok{\}}
\NormalTok{transformed parameters\{}
  \NormalTok{real mu[N];}
  \NormalTok{vector[N] climEff;}
  \NormalTok{vector[N] crowdEff;}
  \NormalTok{climEff <-}\StringTok{ }\NormalTok{C*b2;}
  \NormalTok{crowdEff <-}\StringTok{ }\NormalTok{W*w;}
  \NormalTok{for(n in }\DecValTok{1}\NormalTok{:N)\{}
    \NormalTok{mu[n] <-}\StringTok{ }\KeywordTok{inv_logit}\NormalTok{(a[yid[n]] +}\StringTok{ }\NormalTok{gint[gid[n]] +}\StringTok{ }\NormalTok{b1[yid[n]]*X[n] +}\StringTok{ }
\StringTok{                         }\NormalTok{crowdEff[n] +}\StringTok{ }\NormalTok{climEff[n]);}
  \NormalTok{\}}
\NormalTok{\}}
\NormalTok{model\{}
  \NormalTok{/}\ErrorTok{/}\StringTok{ }\NormalTok{Priors}
  \NormalTok{a_mu ~}\StringTok{ }\KeywordTok{uniform}\NormalTok{(-}\DecValTok{300}\NormalTok{,}\DecValTok{300}\NormalTok{);}
  \NormalTok{w ~}\StringTok{ }\KeywordTok{uniform}\NormalTok{(-}\DecValTok{100}\NormalTok{,}\DecValTok{100}\NormalTok{);}
  \NormalTok{b1_mu ~}\StringTok{ }\KeywordTok{uniform}\NormalTok{(-}\DecValTok{100}\NormalTok{,}\DecValTok{100}\NormalTok{);}
  \NormalTok{sig_a ~}\StringTok{ }\KeywordTok{cauchy}\NormalTok{(}\DecValTok{0}\NormalTok{,}\DecValTok{5}\NormalTok{);}
  \NormalTok{sig_b1 ~}\StringTok{ }\KeywordTok{cauchy}\NormalTok{(}\DecValTok{0}\NormalTok{,}\DecValTok{5}\NormalTok{);}
  \NormalTok{sig_G ~}\StringTok{ }\KeywordTok{cauchy}\NormalTok{(}\DecValTok{0}\NormalTok{,}\DecValTok{5}\NormalTok{);}
  \NormalTok{gint ~}\StringTok{ }\KeywordTok{normal}\NormalTok{(}\DecValTok{0}\NormalTok{, sig_G);}
  \NormalTok{b2 ~}\StringTok{ }\KeywordTok{uniform}\NormalTok{(-}\DecValTok{10}\NormalTok{,}\DecValTok{10}\NormalTok{);}
  \NormalTok{a ~}\StringTok{ }\KeywordTok{normal}\NormalTok{(a_mu, sig_a);}
  \NormalTok{b1 ~}\StringTok{ }\KeywordTok{normal}\NormalTok{(b1_mu, sig_b1);}

  \NormalTok{/}\ErrorTok{/}\StringTok{ }\NormalTok{Likelihood}
  \NormalTok{Y ~}\StringTok{ }\KeywordTok{binomial}\NormalTok{(}\DecValTok{1}\NormalTok{,mu);}
\NormalTok{\}}
\end{Highlighting}
\end{Shaded}

\subsection{Growth (IPM)}\label{growth-ipm}

We modeled growth as a Gaussian process describing genet size at time
$t+1$ as a function of size at $t$ and climate covariates:

\begin{align}
x_{ijQ,t+1} &= \gamma^{G}_{j,t} + \phi^{G}_{jQ} + \beta^{G}_{j,t}x_{ij,t} + \omega^{G}_{j}w_{ij,t} + \nu^{S}_{j}w_{ij,t}x_{ij,t} + \theta^{G}_{jk}C_{k,t} \\
y^{G}_{ijQ,t} &\sim \text{Normal}(x_{ijQ,t+1}, \varepsilon_{ij,t})
\end{align}

where $x$ is log genet size and all other parameters are as described
for the survival regression. We capture non-constant error variance in
growth by modeling the variance around the growth regression
($\varepsilon$) as a nonlinear function of predicted genet size:

\begin{align}
\varepsilon_{ij,t} = a e^{b x_{ijQ,t+1}}
\end{align}

\textbf{Data, process, and parameter models.}

\begin{align}
&\mathtt{data} \qquad &\textbf{y} \sim \text{Normal}(\boldsymbol{\mu}, \boldsymbol{\sigma}^2) \\
&\mathtt{process} \qquad &\mu_{iQ,t+1} = \gamma_{t} + \phi_{Q} + \beta_{t}x_{i,t} + \omega w_{i,t} + \nu w_{i,t}x_{i,t} + \theta_{k}C_{k,t} \\
& \qquad & \sigma^2_{iQ,t} = ae^{b\mu_{iQ,t+1}} \\
&\mathtt{parameters} \qquad &\gamma_{t} \sim \text{Normal}(\tilde{\gamma}, \sigma_\gamma^2)\\
& \qquad & \phi_{Q} \sim \text{Normal}(0,\sigma_Q^2) \\
& \qquad & \beta_{t} \sim \text{Normal}(\tilde{\beta}, \sigma_\beta^2) \\
& \qquad & \omega \sim \text{Normal}(0, 0.001) \\
& \qquad & \nu \sim \text{Normal}(0, 0.001) \\
& \qquad & \boldsymbol{\theta} \sim \text{Normal}(0, 0.001) \\
& \qquad & \tilde{\gamma} \sim \text{Normal}(0, 0.001) \\
& \qquad & \tilde{\beta} \sim \text{Normal}(0, 0.001) \\
& \qquad & a \sim \text{Normal}(0, 0.001) \\
& \qquad & b \sim \text{Normal}(0, 0.001) \\
& \qquad & \sigma_Q^2 \sim \text{Uniform}(0, 1000) \\
& \qquad & \sigma_\gamma^2 \sim \text{Uniform}(0, 1000) \\
& \qquad & \sigma_\beta^2 \sim \text{Uniform}(0, 1000)
\end{align}

\textbf{Full expression of posterior and joint distributions}

\begin{align}
[\boldsymbol{\gamma}, \tilde{\gamma}, \boldsymbol{\phi}, \boldsymbol{\beta}, \tilde{\beta}, \omega, \nu, \boldsymbol{\theta}, a, b, \sigma_Q^2, \sigma_\beta^2] &\propto \\
&\prod_{t=1}^T \prod_{i=1}^n [y_{iQ,t} | \gamma_{t}, \phi_{Q}, \beta_{t}, \omega, \nu, \boldsymbol{\theta}, a, b] [\gamma_{t} | \tilde{\gamma}] [\beta_{t} | \tilde{\beta}, \sigma_\beta^2] \times \\
&\prod_{Q=1}^{Q_{tot}} [\phi_{Q} | \sigma_Q^2] \times \\
&[\boldsymbol{\theta}] [\omega] [\nu] [\tilde{\gamma}] [\tilde{\beta}] [\sigma_Q^2] [\sigma_\beta^2] [a] [b]
\end{align}

\textbf{Stan code for model}

\begin{Shaded}
\begin{Highlighting}[]
\NormalTok{data\{}
  \NormalTok{int<lower=}\DecValTok{0}\NormalTok{>}\StringTok{ }\NormalTok{N; /}\ErrorTok{/}\StringTok{ }\NormalTok{observations}
  \NormalTok{int<lower=}\DecValTok{0}\NormalTok{>}\StringTok{ }\NormalTok{Yrs; /}\ErrorTok{/}\StringTok{ }\NormalTok{years}
  \NormalTok{int<lower=}\DecValTok{0}\NormalTok{>}\StringTok{ }\NormalTok{yid[N]; /}\ErrorTok{/}\StringTok{ }\NormalTok{year id}
  \NormalTok{int<lower=}\DecValTok{0}\NormalTok{>}\StringTok{ }\NormalTok{Covs; /}\ErrorTok{/}\StringTok{ }\NormalTok{climate covariates}
  \NormalTok{int<lower=}\DecValTok{0}\NormalTok{>}\StringTok{ }\NormalTok{G; /}\ErrorTok{/}\StringTok{ }\NormalTok{groups}
  \NormalTok{int<lower=}\DecValTok{0}\NormalTok{>}\StringTok{ }\NormalTok{gid[N]; /}\ErrorTok{/}\StringTok{ }\NormalTok{group id}
  \NormalTok{vector[N] Y; /}\ErrorTok{/}\StringTok{ }\NormalTok{observation vector}
  \NormalTok{matrix[N,Covs] C; /}\ErrorTok{/}\StringTok{ }\NormalTok{climate matrix}
  \NormalTok{vector[N] X; /}\ErrorTok{/}\StringTok{ }\NormalTok{size vector}
  \NormalTok{matrix[N,}\DecValTok{2}\NormalTok{] W; /}\ErrorTok{/}\StringTok{ }\NormalTok{crowding matrix}
\NormalTok{\}}
\NormalTok{parameters\{}
  \NormalTok{real a_mu;}
  \NormalTok{vector[Yrs] a;}
  \NormalTok{real b1_mu;}
  \NormalTok{vector[Yrs] b1;}
  \NormalTok{vector[Covs] b2;}
  \NormalTok{vector[}\DecValTok{2}\NormalTok{] w;}
  \NormalTok{real gint[G];}
  \NormalTok{real tau;}
  \NormalTok{real tauSize;}
  \NormalTok{real<lower=}\DecValTok{0}\NormalTok{>}\StringTok{ }\NormalTok{sig_a;}
  \NormalTok{real<lower=}\DecValTok{0}\NormalTok{>}\StringTok{ }\NormalTok{sig_b1;}
  \NormalTok{real<lower=}\DecValTok{0}\NormalTok{>}\StringTok{ }\NormalTok{sig_G;}
\NormalTok{\}}
\NormalTok{transformed parameters\{}
  \NormalTok{real mu[N];}
  \NormalTok{real<lower=}\DecValTok{0}\NormalTok{>}\StringTok{ }\NormalTok{sigma[N];}
  \NormalTok{vector[N] climEff;}
  \NormalTok{vector[N] crowdEff;}
  \NormalTok{climEff <-}\StringTok{ }\NormalTok{C*b2;}
  \NormalTok{crowdEff <-}\StringTok{ }\NormalTok{W*w;}
  \NormalTok{for(n in }\DecValTok{1}\NormalTok{:N)\{}
    \NormalTok{mu[n] <-}\StringTok{ }\NormalTok{a[yid[n]] +}\StringTok{ }\NormalTok{gint[gid[n]] +}\StringTok{ }\NormalTok{b1[yid[n]]*X[n] +}\StringTok{ }\NormalTok{crowdEff[n] +}\StringTok{ }\NormalTok{climEff[n];}
    \NormalTok{sigma[n] <-}\StringTok{ }\KeywordTok{sqrt}\NormalTok{((}\KeywordTok{fmax}\NormalTok{(tau*}\KeywordTok{exp}\NormalTok{(tauSize*mu[n]), }\FloatTok{0.0000001}\NormalTok{)));  }
  \NormalTok{\}}
\NormalTok{\}}
\NormalTok{model\{}
  \NormalTok{/}\ErrorTok{/}\StringTok{ }\NormalTok{Priors}
  \NormalTok{a_mu ~}\StringTok{ }\KeywordTok{normal}\NormalTok{(}\DecValTok{0}\NormalTok{,}\DecValTok{1000}\NormalTok{);}
  \NormalTok{w ~}\StringTok{ }\KeywordTok{normal}\NormalTok{(}\DecValTok{0}\NormalTok{,}\DecValTok{1000}\NormalTok{);}
  \NormalTok{b1_mu ~}\StringTok{ }\KeywordTok{normal}\NormalTok{(}\DecValTok{0}\NormalTok{,}\DecValTok{1000}\NormalTok{);}
  \NormalTok{tau ~}\StringTok{ }\KeywordTok{normal}\NormalTok{(}\DecValTok{0}\NormalTok{,}\DecValTok{1000}\NormalTok{);}
  \NormalTok{tauSize ~}\StringTok{ }\KeywordTok{normal}\NormalTok{(}\DecValTok{0}\NormalTok{,}\DecValTok{1000}\NormalTok{);}
  \NormalTok{sig_a ~}\StringTok{ }\KeywordTok{uniform}\NormalTok{(}\DecValTok{0}\NormalTok{,}\DecValTok{1000}\NormalTok{);}
  \NormalTok{sig_b1 ~}\StringTok{ }\KeywordTok{uniform}\NormalTok{(}\DecValTok{0}\NormalTok{,}\DecValTok{1000}\NormalTok{);}
  \NormalTok{sig_G ~}\StringTok{ }\KeywordTok{uniform}\NormalTok{(}\DecValTok{0}\NormalTok{,}\DecValTok{1000}\NormalTok{);}
  \NormalTok{for(g in }\DecValTok{1}\NormalTok{:G)}
      \NormalTok{gint[g] ~}\StringTok{ }\KeywordTok{normal}\NormalTok{(}\DecValTok{0}\NormalTok{, sig_G);}
  \NormalTok{for(c in }\DecValTok{1}\NormalTok{:Covs)}
    \NormalTok{b2[c] ~}\StringTok{ }\KeywordTok{normal}\NormalTok{(}\DecValTok{0}\NormalTok{,}\DecValTok{1000}\NormalTok{);}
  \NormalTok{for(y in }\DecValTok{1}\NormalTok{:Yrs)\{}
    \NormalTok{a[y] ~}\StringTok{ }\KeywordTok{normal}\NormalTok{(a_mu, sig_a);}
    \NormalTok{b1[y] ~}\StringTok{ }\KeywordTok{normal}\NormalTok{(b1_mu, sig_b1);}
  \NormalTok{\}}

  \NormalTok{/}\ErrorTok{/}\StringTok{ }\NormalTok{Likelihood}
  \NormalTok{Y ~}\StringTok{ }\KeywordTok{normal}\NormalTok{(mu, sigma);}
\NormalTok{\}}
\end{Highlighting}
\end{Shaded}

\subsection{Recruitment (IPM)}\label{recruitment-ipm}

Our data allows us to track new recruits, but we cannot assign a
specific parent to new genets. So, for recruitment, we work at the
quadrat level and model the number of new individuals of species $j$ in
quadrat $q$ recruiting at time $t+1$ as a function of quadrat
``effective cover'' ($A'$) in the previous year ($t$). Effective cover
is a mixture of observed cover ($A$) in the focal quadrat ($q$) and the
mean cover across the entire group ($\bar{A}$) of $Q$ quadrats in which
$q$ is located:

\begin{equation}
A'_{jq,t} = p_{j}A_{jq,t} + (1-p_{j})\bar{A}_{jQ,t}
\end{equation}

where $p$ is a mixing fraction between 0 and 1 that is estimated within
the model.

We assume the number of individuals, $y^{R}$, recruiting at time $t+1$
follows a negative binomial distribution:

\begin{equation}
y^{R}_{jq,t+1} \sim \text{NegBin}(\lambda_{jq,t+1},\zeta)
\end{equation}

where $\lambda$ is the mean intensity and $\zeta$ is the size parameter.
We define $\lambda$ as:

\begin{equation}
\lambda_{jq,t+1} = A'_{jq,t}e^{(\gamma^{R}_{j,t} + \phi^{R}_{jQ} + \theta^{R}_{jk}C_{k,t} + \omega^{R}\sqrt{A'_{q,t}})}
\end{equation}

where $A'$ is effective cover ($\text{cm}^2$) of species $j$ in quadrat
$q$ and all other terms are as in the survival and growth regressions.

\textbf{Data, process, and parameter models}

\begin{align}
&\mathtt{data} \qquad &\textbf{y} \sim \text{NegBin}(\boldsymbol{\lambda},\zeta) \\
&\mathtt{process} \qquad &\lambda_{jq,t+1} = A'_{q,t}e^{(\gamma_{t} + \phi_{Q} + \theta_{k}C_{k,t} + \omega\sqrt{A'_{q,t}})} \\
&\mathtt{parameters} \qquad &\gamma_{t} \sim \text{Normal}(\tilde{\gamma}, \sigma_\gamma^2)\\
& \qquad & \phi_{Q} \sim \text{Normal}(0,\sigma_Q^2) \\
& \qquad & \omega \sim \text{Uniform}(-100, 100) \\
& \qquad & \boldsymbol{\theta} \sim \text{Uniform}(-10, 10) \\
& \qquad & \tilde{\gamma} \sim \text{Uniform}(-300, 300) \\
& \qquad & \sigma_Q^2 \sim \text{Cauchy}(0, 5) \\
& \qquad & \sigma_\gamma^2 \sim \text{Cauchy}(0, 5) \\
& \qquad & \zeta \sim \text{Uniform}(0, 100) \\
& \qquad & u \sim \text{Uniform}(0, 1)
\end{align}

\textbf{Full expression of posterior and joint distributions}

\begin{align}
[\boldsymbol{\gamma}, \tilde{\gamma}, \boldsymbol{\phi}, \omega, \boldsymbol{\theta}, \sigma_Q^2, \zeta, u] &\propto \\
&\prod_{t=1}^T \prod_{i=1}^n [y_{iQ,t} | \gamma_{t}, \phi_{Q}, \omega, \boldsymbol{\theta}, \zeta, u] [\gamma_{t} | \tilde{\gamma}] \times \\
&\prod_{Q=1}^{Q_{tot}} [\phi_{Q} | \sigma_Q^2] [\boldsymbol{\theta}] [\omega] [\tilde{\gamma}] [\sigma_Q^2] [\zeta] [u]
\end{align}

\textbf{Stan code for model}

\begin{Shaded}
\begin{Highlighting}[]
\NormalTok{data\{}
  \NormalTok{int<lower=}\DecValTok{0}\NormalTok{>}\StringTok{ }\NormalTok{N; /}\ErrorTok{/}\StringTok{ }\NormalTok{observations}
  \NormalTok{int<lower=}\DecValTok{0}\NormalTok{>}\StringTok{ }\NormalTok{Yrs; /}\ErrorTok{/}\StringTok{ }\NormalTok{years}
  \NormalTok{int<lower=}\DecValTok{0}\NormalTok{>}\StringTok{ }\NormalTok{yid[N]; /}\ErrorTok{/}\StringTok{ }\NormalTok{year id}
  \NormalTok{int<lower=}\DecValTok{0}\NormalTok{>}\StringTok{ }\NormalTok{Covs; /}\ErrorTok{/}\StringTok{ }\NormalTok{climate covariates}
  \NormalTok{int<lower=}\DecValTok{0}\NormalTok{>}\StringTok{ }\NormalTok{G; /}\ErrorTok{/}\StringTok{ }\NormalTok{groups}
  \NormalTok{int<lower=}\DecValTok{0}\NormalTok{>}\StringTok{ }\NormalTok{gid[N]; /}\ErrorTok{/}\StringTok{ }\NormalTok{group id}
  \NormalTok{int<lower=}\DecValTok{0}\NormalTok{>}\StringTok{ }\NormalTok{Y[N]; /}\ErrorTok{/}\StringTok{ }\NormalTok{observation vector}
  \NormalTok{matrix[N,Covs] C; /}\ErrorTok{/}\StringTok{ }\NormalTok{climate matrix}
  \NormalTok{vector[N] parents1; /}\ErrorTok{/}\StringTok{ }\NormalTok{crowding vector}
  \NormalTok{vector[N] parents2; /}\ErrorTok{/}\StringTok{ }\NormalTok{crowding vector}
\NormalTok{\}}
\NormalTok{parameters\{}
  \NormalTok{real a_mu;}
  \NormalTok{vector[Yrs] a;}
  \NormalTok{vector[Covs] b2;}
  \NormalTok{real dd;}
  \NormalTok{real gint[G];}
  \NormalTok{real<lower=}\DecValTok{0}\NormalTok{>}\StringTok{ }\NormalTok{sig_a;}
  \NormalTok{real<lower=}\DecValTok{0}\NormalTok{>}\StringTok{ }\NormalTok{theta;}
  \NormalTok{real<lower=}\DecValTok{0}\NormalTok{>}\StringTok{ }\NormalTok{sig_G;}
  \NormalTok{real<lower=}\DecValTok{0}\NormalTok{, upper=}\DecValTok{1}\NormalTok{>}\StringTok{ }\NormalTok{u;}
\NormalTok{\}}
\NormalTok{transformed parameters\{}
  \NormalTok{real mu[N];}
  \NormalTok{vector[N] climEff;}
  \NormalTok{vector[N] trueP1;}
  \NormalTok{vector[N] trueP2;}
  \NormalTok{vector[N] lambda;}
  \NormalTok{vector[N] q;}
  \NormalTok{climEff <-}\StringTok{ }\NormalTok{C*b2;}
  \NormalTok{for(n in }\DecValTok{1}\NormalTok{:N)\{}
    \NormalTok{trueP1[n] <-}\StringTok{ }\NormalTok{parents1[n]*u +}\StringTok{ }\NormalTok{parents2[n]*(}\DecValTok{1}\NormalTok{-u);}
    \NormalTok{trueP2[n] <-}\StringTok{ }\KeywordTok{sqrt}\NormalTok{(trueP1[n]);}
    \NormalTok{mu[n] <-}\StringTok{ }\KeywordTok{exp}\NormalTok{(a[yid[n]] +}\StringTok{ }\NormalTok{gint[gid[n]] +}\StringTok{ }\NormalTok{dd*trueP2[n] +}\StringTok{ }\NormalTok{climEff[n]);}
    \NormalTok{lambda[n] <-}\StringTok{ }\NormalTok{trueP1[n]*mu[n];}
    \NormalTok{q[n] <-}\StringTok{ }\NormalTok{lambda[n]*theta;}
  \NormalTok{\}}
\NormalTok{\}}
\NormalTok{model\{}
  \NormalTok{/}\ErrorTok{/}\StringTok{ }\NormalTok{Priors}
  \NormalTok{u ~}\StringTok{ }\KeywordTok{uniform}\NormalTok{(}\DecValTok{0}\NormalTok{,}\DecValTok{1}\NormalTok{);}
  \NormalTok{theta ~}\StringTok{ }\KeywordTok{uniform}\NormalTok{(}\DecValTok{0}\NormalTok{,}\DecValTok{100}\NormalTok{);}
  \NormalTok{a_mu ~}\StringTok{ }\KeywordTok{normal}\NormalTok{(}\DecValTok{0}\NormalTok{,}\DecValTok{1000}\NormalTok{);}
  \NormalTok{dd ~}\StringTok{ }\KeywordTok{uniform}\NormalTok{(-}\DecValTok{100}\NormalTok{,}\DecValTok{100}\NormalTok{);}
  \NormalTok{sig_a ~}\StringTok{ }\KeywordTok{cauchy}\NormalTok{(}\DecValTok{0}\NormalTok{,}\DecValTok{5}\NormalTok{);}
  \NormalTok{sig_G ~}\StringTok{ }\KeywordTok{cauchy}\NormalTok{(}\DecValTok{0}\NormalTok{,}\DecValTok{5}\NormalTok{);}
  \NormalTok{for(g in }\DecValTok{1}\NormalTok{:G)}
      \NormalTok{gint[g] ~}\StringTok{ }\KeywordTok{normal}\NormalTok{(}\DecValTok{0}\NormalTok{, sig_G);}
  \NormalTok{for(c in }\DecValTok{1}\NormalTok{:Covs)}
    \NormalTok{b2 ~}\StringTok{ }\KeywordTok{uniform}\NormalTok{(-}\DecValTok{100}\NormalTok{,}\DecValTok{100}\NormalTok{);}
  \NormalTok{for(y in }\DecValTok{1}\NormalTok{:Yrs)\{}
    \NormalTok{a[y] ~}\StringTok{ }\KeywordTok{normal}\NormalTok{(a_mu, sig_a);}
  \NormalTok{\}}

  \NormalTok{/}\ErrorTok{/}\StringTok{ }\NormalTok{Likelihood}
  \NormalTok{Y ~}\StringTok{ }\KeywordTok{neg_binomial_2}\NormalTok{(q, theta);}
\NormalTok{\}}
\end{Highlighting}
\end{Shaded}

\subsection{Quadrat based model (QBM)}\label{quadrat-based-model-qbm}

The model for quadrat cover change from time $t$ to $t+1$ is

\begin{align}
x_{jq,t+1} &= \gamma^{P}_{j,t} + \phi^{P}_{jQ} + \beta^{P}_{j,t}x_{jq,t} + \theta^{P}_{jk}C_{k,t} \\
y^{P}_{jq,t+1} &\sim \text{LogNormal}(x_{jq,t+1}, \tau_{j}) \text{T}[0,1]
\end{align}

where $x_{jq,t}$ is the log of species' $j$ proportional cover in
quadrat $q$ at time $t$ and all other parameters are as in the
individual-level growth model (Eq. 3). Again, note that the climate
covariate vector (\textbf{C}) includes the climate$\times$cover
interaction. The log normal likelihood includes a truncation
(T{[}0,1{]}) to ensure that predicted values do not exceed 100\% cover.

\textbf{Data, process, and parameter models.}

\begin{align}
&\mathtt{data} \qquad &\textbf{y} \sim \text{LogNormal}(\boldsymbol{\mu}, \boldsymbol{\tau}) \text{T}[0,1] \\
&\mathtt{process} \qquad &\mu_{iQ,t+1} = \gamma_{t} + \phi_{Q} + \beta_{t}x_{i,t} + \theta_{k}C_{k,t} \\
&\mathtt{parameters} \qquad &\gamma_{t} \sim \text{Normal}(\tilde{\gamma}, \sigma_\gamma^2)\\
& \qquad & \phi_{Q} \sim \text{Normal}(0,\sigma_Q^2) \\
& \qquad & \beta_{t} \sim \text{Normal}(\tilde{\beta}, \sigma_\beta^2) \\
& \qquad & \boldsymbol{\theta} \sim \text{Uniform}(-10, 10) \\
& \qquad & \tilde{\gamma} \sim \text{Uniform}(-300, 300) \\
& \qquad & \tilde{\beta} \sim \text{Uniform}(-100, 100) \\
& \qquad & \sigma_Q^2 \sim \text{Cauchy}(0, 5) \\
& \qquad & \sigma_\gamma^2 \sim \text{Cauchy}(0, 5) \\
& \qquad & \sigma_\beta^2 \sim \text{Cauchy}(0, 5) \\
& \qquad & \tau \sim \text{Cauchy}(0, 5)
\end{align}

\textbf{Full expression of posterior and joint distributions}

\begin{align}
[\boldsymbol{\gamma}, \tilde{\gamma}, \boldsymbol{\phi}, \boldsymbol{\beta}, \tilde{\beta}, \boldsymbol{\theta}, \sigma_Q^2, \sigma_\beta^2, \tau] &\propto \\
&\prod_{t=1}^T \prod_{i=1}^n [y_{iQ,t} | \gamma_{t}, \phi_{Q}, \beta_{t}, \boldsymbol{\theta}, \tau] [\gamma_{t} | \tilde{\gamma}] [\beta_{t} | \tilde{\beta}, \sigma_\beta^2] \times \\
&\prod_{Q=1}^{Q_{tot}} [\phi_{Q} | \sigma_Q^2] \times \\
&[\boldsymbol{\theta}] [\omega] [\tilde{\gamma}] [\tilde{\beta}] [\sigma_Q^2] [\sigma_\beta^2] [\tau]
\end{align}

\textbf{Stan code for model}

\begin{Shaded}
\begin{Highlighting}[]
\NormalTok{data\{}
  \NormalTok{int<lower=}\DecValTok{0}\NormalTok{>}\StringTok{ }\NormalTok{N; /}\ErrorTok{/}\StringTok{ }\NormalTok{observations}
  \NormalTok{int<lower=}\DecValTok{0}\NormalTok{>}\StringTok{ }\NormalTok{Yrs; /}\ErrorTok{/}\StringTok{ }\NormalTok{years}
  \NormalTok{int<lower=}\DecValTok{0}\NormalTok{>}\StringTok{ }\NormalTok{yid[N]; /}\ErrorTok{/}\StringTok{ }\NormalTok{year id}
  \NormalTok{int<lower=}\DecValTok{0}\NormalTok{>}\StringTok{ }\NormalTok{Covs; /}\ErrorTok{/}\StringTok{ }\NormalTok{climate covariates}
  \NormalTok{int<lower=}\DecValTok{0}\NormalTok{>}\StringTok{ }\NormalTok{G; /}\ErrorTok{/}\StringTok{ }\NormalTok{groups}
  \NormalTok{int<lower=}\DecValTok{0}\NormalTok{>}\StringTok{ }\NormalTok{gid[N]; /}\ErrorTok{/}\StringTok{ }\NormalTok{group id}
  \NormalTok{real<lower=}\DecValTok{0}\NormalTok{,upper=}\DecValTok{1}\NormalTok{>}\StringTok{ }\NormalTok{Y[N]; /}\ErrorTok{/}\StringTok{ }\NormalTok{observation vector}
  \NormalTok{matrix[N,Covs] C; /}\ErrorTok{/}\StringTok{ }\NormalTok{climate matrix}
  \NormalTok{vector[N] X; /}\ErrorTok{/}\StringTok{ }\NormalTok{size vector}
\NormalTok{\}}
\NormalTok{parameters\{}
  \NormalTok{real a_mu;}
  \NormalTok{vector[Yrs] a;}
  \NormalTok{real b1_mu;}
  \NormalTok{vector[Yrs] b1;}
  \NormalTok{vector[Covs] b2;}
  \NormalTok{vector[G] gint;}
  \NormalTok{real<lower=}\DecValTok{0}\NormalTok{>}\StringTok{ }\NormalTok{sig_a;}
  \NormalTok{real<lower=}\DecValTok{0}\NormalTok{>}\StringTok{ }\NormalTok{sig_b1;}
  \NormalTok{real<lower=}\DecValTok{0}\NormalTok{>}\StringTok{ }\NormalTok{sig_G;}
  \NormalTok{real<lower=}\DecValTok{0}\NormalTok{>}\StringTok{ }\NormalTok{tau;}
\NormalTok{\}}
\NormalTok{transformed parameters\{}
  \NormalTok{real mu[N];}
  \NormalTok{vector[N] climEff;}
  \NormalTok{climEff <-}\StringTok{ }\NormalTok{C*b2;}
  \NormalTok{for(n in }\DecValTok{1}\NormalTok{:N)}
    \NormalTok{mu[n] <-}\StringTok{ }\NormalTok{a[yid[n]] +}\StringTok{ }\NormalTok{gint[gid[n]] +}\StringTok{ }\NormalTok{b1[yid[n]]*X[n] +}\StringTok{ }\NormalTok{climEff[n];}
\NormalTok{\}}
\NormalTok{model\{}
  \NormalTok{/}\ErrorTok{/}\StringTok{ }\NormalTok{Priors}
  \NormalTok{a_mu ~}\StringTok{ }\KeywordTok{uniform}\NormalTok{(-}\DecValTok{300}\NormalTok{,}\DecValTok{300}\NormalTok{);}
  \NormalTok{b1_mu ~}\StringTok{ }\KeywordTok{uniform}\NormalTok{(-}\DecValTok{100}\NormalTok{,}\DecValTok{100}\NormalTok{);}
  \NormalTok{sig_a ~}\StringTok{ }\KeywordTok{cauchy}\NormalTok{(}\DecValTok{0}\NormalTok{,}\DecValTok{5}\NormalTok{);}
  \NormalTok{sig_b1 ~}\StringTok{ }\KeywordTok{cauchy}\NormalTok{(}\DecValTok{0}\NormalTok{,}\DecValTok{5}\NormalTok{);}
  \NormalTok{sig_G ~}\StringTok{ }\KeywordTok{cauchy}\NormalTok{(}\DecValTok{0}\NormalTok{,}\DecValTok{5}\NormalTok{);}
  \NormalTok{gint ~}\StringTok{ }\KeywordTok{normal}\NormalTok{(}\DecValTok{0}\NormalTok{, sig_G);}
  \NormalTok{b2 ~}\StringTok{ }\KeywordTok{uniform}\NormalTok{(-}\DecValTok{10}\NormalTok{,}\DecValTok{10}\NormalTok{);}
  \NormalTok{a ~}\StringTok{ }\KeywordTok{normal}\NormalTok{(a_mu, sig_a);}
  \NormalTok{b1 ~}\StringTok{ }\KeywordTok{normal}\NormalTok{(b1_mu, sig_b1);}
  \NormalTok{tau ~}\StringTok{ }\KeywordTok{cauchy}\NormalTok{(}\DecValTok{0}\NormalTok{,}\DecValTok{5}\NormalTok{);}

  \NormalTok{/}\ErrorTok{/}\NormalTok{Likelihood}
  \NormalTok{Y ~}\StringTok{ }\KeywordTok{lognormal}\NormalTok{(mu, tau);}
\NormalTok{\}}
\end{Highlighting}
\end{Shaded}

\end{document}

\documentclass[12pt,]{article}
\usepackage{lmodern}
\usepackage{amssymb,amsmath}
\usepackage{ifxetex,ifluatex}
\usepackage{fixltx2e} % provides \textsubscript
\ifnum 0\ifxetex 1\fi\ifluatex 1\fi=0 % if pdftex
  \usepackage[T1]{fontenc}
  \usepackage[utf8]{inputenc}
\else % if luatex or xelatex
  \ifxetex
    \usepackage{mathspec}
    \usepackage{xltxtra,xunicode}
  \else
    \usepackage{fontspec}
  \fi
  \defaultfontfeatures{Mapping=tex-text,Scale=MatchLowercase}
  \newcommand{\euro}{€}
\fi
% use upquote if available, for straight quotes in verbatim environments
\IfFileExists{upquote.sty}{\usepackage{upquote}}{}
% use microtype if available
\IfFileExists{microtype.sty}{%
\usepackage{microtype}
\UseMicrotypeSet[protrusion]{basicmath} % disable protrusion for tt fonts
}{}
\usepackage[margin=1in]{geometry}
\ifxetex
  \usepackage[setpagesize=false, % page size defined by xetex
              unicode=false, % unicode breaks when used with xetex
              xetex]{hyperref}
\else
  \usepackage[unicode=true]{hyperref}
\fi
\hypersetup{breaklinks=true,
            bookmarks=true,
            pdfauthor={},
            pdftitle={},
            colorlinks=true,
            citecolor=blue,
            urlcolor=blue,
            linkcolor=magenta,
            pdfborder={0 0 0}}
\urlstyle{same}  % don't use monospace font for urls
\setlength{\parindent}{0pt}
\setlength{\parskip}{6pt plus 2pt minus 1pt}
\setlength{\emergencystretch}{3em}  % prevent overfull lines
\setcounter{secnumdepth}{0}

%%% Use protect on footnotes to avoid problems with footnotes in titles
\let\rmarkdownfootnote\footnote%
\def\footnote{\protect\rmarkdownfootnote}

%%% Change title format to be more compact
\usepackage{titling}

% Create subtitle command for use in maketitle
\newcommand{\subtitle}[1]{
  \posttitle{
    \begin{center}\large#1\end{center}
    }
}

\setlength{\droptitle}{-2em}
  \title{}
  \pretitle{\vspace{\droptitle}}
  \posttitle{}
  \author{}
  \preauthor{}\postauthor{}
  \date{}
  \predate{}\postdate{}

\usepackage{lineno}
\linenumbers
\usepackage{setspace}
\doublespacing
\usepackage{todonotes}
\usepackage[document]{ragged2e}
\pagewiselinenumbers


\begin{document}

\maketitle


\subsubsection{Title}\label{title}

Do we need demographic data to forecast population responses to climate
change?

\subsubsection{Authors}\label{authors}

\begin{enumerate}
\def\labelenumi{\arabic{enumi}.}
\itemsep1pt\parskip0pt\parsep0pt
\item
  Andrew T. Tredennick (corresponding author:
  \href{mailto:atredenn@gmail.com}{\nolinkurl{atredenn@gmail.com}})
\item
  Peter B. Adler
  (\href{mailto:peter.adler@usu.edu}{\nolinkurl{peter.adler@usu.edu}})
\end{enumerate}

\subsubsection{Affiliations}\label{affiliations}

\begin{enumerate}
\def\labelenumi{\arabic{enumi}.}
\itemsep1pt\parskip0pt\parsep0pt
\item
  Department of Wildland Resources and the Ecology Center, 5230 Old Main
  Hill, Utah State University, Logan, Utah 84322-5230 USA
\item
  Department of Wildland Resources and the Ecology Center, 5230 Old Main
  Hill, Utah State University, Logan, Utah 84322-5230 USA
\end{enumerate}

\newpage{}

\subsection{Summary}\label{summary}

\begin{enumerate}
\def\labelenumi{\arabic{enumi}.}
\itemsep1pt\parskip0pt\parsep0pt
\item
  Rapid climate change has generated growing interest in forecasts of
  future population trajectories. Traditional population models,
  typically built using detailed demographic observations from one study
  site, can address climate change impacts at one location, but are
  difficult to scale up to the landscape and regional scales relevant to
  management decisions. An alternative is to build models using
  population-level data that is much easier to collect over broad
  spatial scales than individual-level data. However, such models ignore
  the fact that climate drives population growth through its influence
  on individual performance.
\item
  We test the consequences of aggregating individual responses when
  forecasting climate change impacts on two perennial grass species
  (\emph{Pascopyrum smithii} and \emph{Sprobolis flexuosus}). We
  parameterized two population models for each species, one based on
  individual-level data (survival, growth and recruitment) and one on
  population-level data (percent cover), and compared their accuracy,
  precision, and sensitivity to climate variables. For both models we
  used Bayesian ridge regression to identify the optimal predictive
  model in terms of climate covariate strengths.
\item
  The individual-level model was more accurate and precise than the
  aggregated model when predicting out-of-sample observations. When
  comparing climate effects from both models, the population-level model
  missed important climate effects from at least one vital rate for each
  species. Increasing the sample size at the population-level would not
  necessarily reduce forecast uncertainty; the way to reduce uncertainty
  is to capture unique climate dependence of individual vital rates.
\item
  Our analysis indicates that there is no shortcut to forecasting
  climate change impacts on plant populations --- detailed demographic
  data is essential. Despite the superiority of the individual-level
  model, the forecasts it generated still were too uncertain to be
  useful for decision-makers. We need new methods to collect demographic
  data efficiently across environmental gradients in space and time.
\end{enumerate}

\emph{Key words: forecasting, climate change, grassland, integral
projection model, population model}

\newpage{}

\subsection{Introduction}\label{introduction}

Perhaps the greatest challenge for ecology in the 21st century is to
forecast the impacts of environmental change (Clark et al. 2001, Petchey
et al. 2015). Forecasts require sophisticated modeling approaches that
fully account for uncertainty and variability in both ecological process
and model parameters (Luo et al. 2011, but see Perretti et al. 2013 for
an argument against modeling the ecological process). The increasing
statistical sophistication of population models (Rees and Ellner 2009)
makes them promising tools for predicting the impacts of environmental
change on species persistence and abundance. But reconciling the scales
at which population models are parameterized and the scales at which
environmental changes play out remains a challenge (Clark et al. 2010,
2012, Freckleton et al. 2011, Queenborough et al. 2011). The problem is
that most population models are built using data from a single study
site because collecting those data, which involves tracking the fates of
individuals plants, is so difficult. The resulting models cannot be
applied to the landscape and regional scales relevant to decision-making
without information about how the fitted parameters respond to spatial
variation in biotic and abiotic drivers (S{æ}ther et al. 2007). The
limited spatial extent of individual-level demographic datasets
constrains our ability to use population models to address applied
questions about the consequences of climate change.

The inability of most population models to address landscape-scale
problems may explain why land managers and conservation planners have
embraced species distribution models (SDMs) (see Guisan and Thuiller
2005 for a review). SDMs typically rely on easy-to-collect
presence/absence data (but see Clark et al. 2014 for new methods) and
remotely-sensed environmental covariates that allow researchers to model
large spatial extents (e.g., Maiorano et al. 2013). Thus, it is
relatively straightforward to parameterize and project SDMs over
landscapes and regions. However, the limitations of SDMs are well known
(Pearson and Dawson 2003, Elith and Leathwick 2009, Ara{ú}jo and
Peterson 2012). Ideally, researchers would provide managers with
landscape-scale population models, combining the extent of SDMs with
information about dynamics and species abundances (Schurr et al. 2012,
Merow et al. 2014).

Aggregate measures of population status, rather than individual
performance, offer an intriguing alternative for modeling populations
(Clark and Bj{ø}rnstad 2004, Freckleton et al. 2011). Population-level
data cannot provide inference about demographic mechanisms, but might be
sufficient for modeling future population states, especially since such
data are feasible to collect across broad spatial extents (e.g.,
Queenborough et al. 2011). The choice between individual and
population-level data involves a difficult trade-off: while
individual-level data leads to more mechanistic models, population-level
data leads to models that can be applied over greater spatial and
temporal extents. An open question is how much forecasting skill is lost
when we build models based on population rather than individual-level
data.

To date, most empirical population modelers have relied on
individual-level data, with few attempts to capitalize on
population-level measures. An important exception was an effort by
Taylor and Hastings (2004) to model the population growth rate of an
invasive species to identify the best strategies for invasion control.
They used a ``density-structured'' model where the state variable is a
discrete density state rather than a continuous density measure. Such
models do not require individual-level demographic data and can
adequately describe population dynamics. Building on Taylor and Hastings
(2004), Freckleton et al. (2011) showed that density-structured models
compare well to continuous models in theory, and Queenborough et al.
(2011) provide empirical evidence that density-structured models are
capable of reproducing population dynamics at landscape spatial scales,
even if some precision is lost when compared to fully continuous models.
The appeal of density-structured approaches is clear. However, none of
these models included environmental covariates.

Addressing climate change questions with models fit to population-level
data is potentially problematic. It is individuals, not populations,
that respond to climate variables (Clark et al. 2012). Ignoring this
fact amounts to an ``ecological fallacy'', where inference about the
individual relies on statistical inference on the group (Piantadosi et
al. 1988). Population growth (or decline) is the outcome of demographic
processes such as survival, growth, and recruitment that occur at the
level of individual plants. Climate can affect each demographic process
in unique, potentially opossing, ways (Dalgleish et al. 2011). These
unique climate responses may be difficult to resolve in statistical
models based on population-level data where demographic processes are
not identifiable. If important climate effects are missed because of the
aggregation inherent in in population-level data, then population models
built with such data will make uninformative or unreliable forecasts.

Here, we compare the forecasting skill of statistical and population
models based on aggregated, population-level data with models based on
individual-level data. We used a unique demographic dataset that tracks
the fates of individual plants from four species over 14 years to build
two kinds of single-species population models, traditional models using
individual growth, survival, and recruitment data and alternative models
based on basal cover. In both models, interannual variation is
explained, in part, by climate covariates. We first quantify forecasting
skill using cross-validation. We then performed simulations to quantify
the sensitivities of species' cover to small perturbations in average
precipitation and temperature. Based on the cross-validation results,
predictions of individual level models were clearly better, but,
unfortunately, still too uncertain to inform management decisions.

\subsection{Materials and Methods}\label{materials-and-methods}

\subsubsection{Study site and data}\label{study-site-and-data}

Our demographic data come from the Fort Keogh Livestock and Range
Research Laboratory in eastern Montana's northern mixed prairie near
Miles City, Montana, USA (\(46^{\circ}\) 19' N, \(105^{\circ}\) 48' W).
The dataset is freely available on Ecological Archives\footnote{\url{http://esapubs.org/archive/ecol/E092/143/}}
(Anderson et al. 2011) , and interested readers should refer to the
metadata for a complete description. The site is about 800 m above sea
level and mean annual precipitation (1878-2009) is 334 mm, with most
annual precipitation falling from April through September. The community
is grass-dominated and we focused on the four most abundant grass
species: \emph{Bouteloua gracilis} (BOGR), \emph{Hesperostipa comata}
(HECO), \emph{Pascopyrum smithii} (PASM), and \emph{Poa secunda} (POSE)
(Fig. 1).

From 1932 to 1945 individual plants were identified and mapped annually
in 44 1-\(\text{m}^2\) quadrats using a pantograph. The quadrats were
distributed in six pastures, each assigned a grazing treatment of light
(1.24 ha/animal unit month), moderate (0.92 ha/aum), and heavy (0.76
ha/aum) stocking rates (two pastures per treatment). In this analysis we
account for potential differences among the grazing treatments, but do
not focus on grazing\(\times\)climate interactions. The annual maps of
the quadrats were digitized and the fates of individual plants tracked
and extracted using a computer program (Lauenroth and Adler 2008, Chu et
al. 2014). Daily climate data are available for the duration of the data
collection period (1932 - 1945) from the Miles City airport, Wiley
Field, 9 km from the study site.

We modeled each grass population based on two levels of data: individual
and quadrat (Fig. 2). The individual data is the ``raw'' data. For the
quadrat-level we data we simply sum individual basal cover for each
quadrat by species. This is equivalent to a near-perfect census of
quadrat percent cover because previous analysis shows that measurement
error at the individual-level is small (Chu and Adler 2014). Based on
these two datasets we can compare population models built using
individual-level data and aggregated, quadrat-level data.

All R code and data necessary to reproduce our analysis is archived on
GitHub as release v1.0\footnote{\emph{Note to reviewers}: so that v1.0
  will be associated with the published version of the manuscript, we
  have released v0.1 to be associated with this review version.}
(\url{http://github.com/atredennick/MicroMesoForecast/releases}). That
stable release will remain static as a record of this analysis, but
subsequent versions may appear if we update this work. We have also
deposited the v1.0 release on Dryad (\emph{link here after acceptance}).

\subsubsection{Stastical models of vital
rates}\label{stastical-models-of-vital-rates}

At both levels of inference (individual and quadrat), the building
blocks of our population models are vital rate regressions. For
individual-level data, we fit regressions for survival, growth, and
recruitment for each species. At the quadrat-level, we fit a single
regression model for population growth. We describe the statistical
models separately since fitting the models required different
approaches. All models contain five climate covariates that we chose
\emph{a priori}: ``water year'' precipitation at \emph{t}-1 (lagppt);
April through June precipitation at \emph{t}-1 and \emph{t}-2 (ppt1 and
ppt2, respectively) and April through June temperature at \emph{t}-1 and
\emph{t}-2 (TmeanSpr1 and TmeanSpr2, respectively), where \emph{t} is
the observation year. We also include interactions among same-year
climate covariates (e.g., ppt1 \(\times\) TmeansSpr1) and climate
\(\times\) size interactions. Climate \(\times\) size interactions are
for climate main effects only; we do not include interactions between
size and pairs of interacting climate effects.

We fit all models using a hierarchical Bayesian approach. The models are
fully descibed in Appendix A, so here we focus on the main process and
the model likelihood. For the likelihood models, \(\textbf{y}^X\) is
always the relevant vector of observations for vital rate \emph{X}
(\(X = S, G, R, or P\) for survival, growth, recruitment, or population
growth). For example, \(\textbf{y}^S\) is a vector of 0's and 1's
indicating whether a genet survives from \emph{t} to \emph{t+1}, or not.

\paragraph{Vital rate models at the individual
level}\label{vital-rate-models-at-the-individual-level}

We used logistic regression to model survival probability (\(S\)) of
genet \(i\) from species \(j\) in quadrat group \(Q\) from time \(t\) to
\(t+1\):

\begin{align}
\text{logit}(S_{ijQ,t}) &= \gamma^{S}_{j,t} + \phi^{S}_{jQ} + \beta^{S}_{j,t}x_{ij,t} + \omega^{S}_{j}w_{ij,t} + \nu^{S}_{j}w_{ij,t}x_{ij,t} + \theta^{S}_{jk}C_{k,t} \\
y^{S}_{ijQ,t} &\sim \text{Bernoulli}(S_{ijQ,t})
\end{align}

where \(x_{ij,t}\) is the log of genet size, \(\gamma^{S}_{j,t}\) is a
year-specific intercept, \(\beta^{S}_{j,t}\) is the year-specific slope
parameter for size, \(\phi^{S}_{jQ}\) is the random effect of quadrat
group location, and \(\theta^{S}_{k}\) is the fixed parameter for the
effect of the \(k\)th climate covariate at time \(t\) (\(C_{k,t}\)).
Note that the vector of climate covariates (\textbf{C}) includes climate
variable interactions and climate\(\times\)size interactions. We include
density-dependence by estimating the effect of crowding on the focal
individual by other individuals of the same species. \(\omega\) is the
effect of crowding and \(w_{t,Q}\) is the crowding experienced by the
focal individual at time \(t\) in quadrat group \(Q\). We include a
size\(\times\)crowding interaction effect (\(\nu^{S}\)).

We modeled growth as a Gaussian process describing genet size at time
\(t+1\) as a function of size at \(t\) and climate covariates:

\begin{align}
x_{ijQ,t+1} &= \gamma^{G}_{j,t} + \phi^{G}_{jQ} + \beta^{G}_{j,t}x_{ij,t} + \omega^{G}_{j}w_{ij,t} + \nu^{S}_{j}w_{ij,t}x_{ij,t} + \theta^{G}_{jk}C_{k,t} \\
y^{G}_{ijQ,t} &\sim \text{Normal}(x_{ijQ,t+1}, \varepsilon_{ij,t})
\end{align}

where \(x\) is log genet size and all other parameters are as described
for the survival regression. We capture non-constant error variance in
growth by modeling the variance around the growth regression
(\(\varepsilon\)) as a nonlinear function of predicted genet size:

\begin{align}
\varepsilon_{ij,t} = a e^{b x_{ijQ,t+1}}
\end{align}

Our data allows us to track new recruits, but we cannot assign a
specific parent to new genets. THerefore, we model recruitment at the
quadrat level: the number of new individuals of species \(j\) in quadrat
\(q\) recruiting at time \(t+1\) as a function of quadrat ``effective
cover'' (\(A'\)) in the previous year (\(t\)). Effective cover is a
mixture of observed cover (\(A\)) in the focal quadrat (\(q\)) and the
mean cover across the entire group (\(\bar{A}\)) of \(Q\) quadrats in
which \(q\) is located:

\begin{equation}
A'_{jq,t} = p_{j}A_{jq,t} + (1-p_{j})\bar{A}_{jQ,t}
\end{equation}

where \(p\) is a mixing fraction between 0 and 1 that is estimated
within the model.

We assume the number of individuals, \(y^{R}\), recruiting at time
\(t+1\) follows a negative binomial distribution:

\begin{equation}
y^{R}_{jq,t+1} \sim \text{NegBin}(\lambda_{jq,t+1},\zeta)
\end{equation}

where \(\lambda\) is the mean intensity and \(\zeta\) is the size
parameter. We define \(\lambda\) as:

\begin{equation}
\lambda_{jq,t+1} = A'_{jq,t}e^{(\gamma^{R}_{j,t} + \phi^{R}_{jQ} + \theta^{R}_{jk}C_{k,t} + \omega^{R}\sqrt{A'_{q,t}})}
\end{equation}

where \(A'\) is effective cover (\(\text{cm}^2\)) of species \(j\) in
quadrat \(q\) and all other terms are as in the survival and growth
regressions.

\paragraph{Population model at the quadrat
level}\label{population-model-at-the-quadrat-level}

The statistical approach used to model aggregated data depends on the
type of data collected. We have percent cover data, which can easily be
transformed to proportion data. We first considered fitting three vital
rate models analagous to those we fit at the individual level: one for
probability of extirpation within a quadrat (analagous to survival), one
for cover change within a quadrat (analagous to growth), and one for
probability of colonization within a quadrat (analagous to recruitment).
However, within-quadrat extirpation and colonization events were rare in
our time series (\(N=9\) and \(N=10\), respectively, across all
species). Given the broad spatial distribution of the quadrats we are
studying, it is safe to assume that these events are in fact rare enough
to be ignored for our purposes. So we constrained our statistical
modeling of vital rates at the population level to change in percent
cover within quadrats. For the remaining discussion of statistical
modeling, we refer to proportion data, which is simply percent cover
divided by 100.

An obvious choice for fitting a linear model to proportion data is beta
regression because the support of the beta distribution is {[}0,1{]},
not including true zeros or ones. However, when we used fitted model
parameters from a beta regression in a quadrat-based population model,
the simulated population tended toward 100\% cover for all species. We
therefore chose a more constrained modeling approach based on a
truncated log-normal likelihood. The model for quadrat cover change from
time \(t\) to \(t+1\) is

\begin{align}
x_{jq,t+1} &= \gamma^{P}_{j,t} + \phi^{P}_{jQ} + \beta^{P}_{j,t}x_{jq,t} + \theta^{P}_{jk}C_{k,t} \\
y^{P}_{jq,t+1} &\sim \text{LogNormal}(x_{jq,t+1}, \tau_{j}) \text{T}[0,1]
\end{align}

where \(x_{jq,t}\) is the log of species' \(j\) proportional cover in
quadrat \(q\) at time \(t\) and all other parameters are as in the
individual-level growth model (Eq. 3). Again, note that the climate
covariate vector (\textbf{C}) includes the climate\(\times\)cover
interaction. The log normal likelihood includes a truncation
(T{[}0,1{]}) to ensure that predicted values do not exceed 100\% cover.

\subsubsection{Model fitting}\label{model-fitting}

Our Bayesian approach to fitting the vital rate models required choosing
appropriate priors for unknown parameters and deciding which, if any, of
those priors should be hierarchical. We decided to fit models where all
terms were fit by species. Within a species, we fit yearly size effects
and yearly intercepts hierarchically where year-specific coefficients
were drawn from global distributions representing the mean size effect
and intercept. Quadrat random effects were also fit hierarchically, with
quadrat offsets being drawn from distributions with mean zero and a
shared variance term (independent Gaussian priors, Appendix A). Climate
effects were not modeled hierarchically, and each was given a diffuse
prior distribution. We used standard diffuse priors for all unknown
parameters (Appendix A).

All of our analyses (model fitting and simulating) were conducted in R
(R Core Development Team 2013). We used the `No-U-Turn' MCMC sampler in
Stan (Stan Development Team 2014a) to estimate the posterior
distributions of model parameters using the package `rstan' (Stan
Development Team 2014b). We obtained posterior distributions for all
model parameters from three parallel MCMC chains run for 1,000
iterations after discarding an initial 1,000 iterations. Such short MCMC
chains may surprise readers more familiar with other MCMC samplers
(i.e.~JAGS or WinBUGS), but the Stan sampler is exceptionally efficient,
which reduces the number of iterations needed to achieve convergence. We
assessed convergence visually and made sure scale reduction factors for
all parameters were less than 1.01. For the purposes of including
parameter uncertainty in our population models, we saved the final 1,000
iterations from each of the three MCMC chains to be used as randomly
drawn values during population simulation. This step alleviates the need
to reduce model parameters by model selection since sampling from the
full parameter space in the MCMC ensures that if a parameter broadly
overlaps zero, on average the effect in the population models will also
be near zero. We report the posterior mean, standard deviation, and 95\%
Bayesian Credible Intervals for every parameter of each model for each
species in Appendix B.

\subsubsection{Population models}\label{population-models}

With the posterior distribution of the vital rate statistical models in
hand, it is straightforward to simulate the population models. We used
an Integral Projection Model (IPM) to model populations based on
individual-level data (Ellner and Rees 2006) and a quadrat-based version
of an individually-based model (Quadrat-Based Model, QBM) to model
populations based on quadrat-level data. We describe each in turn.

\paragraph{Integral projection model}\label{integral-projection-model}

We use a stochastic IPM (Rees and Ellner 2009) that includes the climate
covariates from the vital rate statistical models. In all simulations we
ignore the random year effects so that interannual variation is driven
solely by climate. We fit the random year effects in the vital rate
regressions to avoid over-attributing variation to climate covariates.
Our IPM follows the specification of Chu and Adler (2015) where the
population of species \emph{j} is a density function \(n(u_{j},t)\)
giving the density of sized-\emph{u} genets at time \emph{t}. Genet size
is on the natural log scale, so that \(n(u_{j},t)du\) is the number of
genets whose area (on the arithmetic scale) is between \(e^{u_{j}}\) and
\(e^{u_{j}+du}\). The density function for any size \emph{v} at time
\(t+1\) is

\begin{equation}
n(v_{j},t+1) = \int_{L_{j}}^{U_{j}} k_{j}(v_{j},u_{j},\bar{w_{j}}(u_{j}))n(u_{j},t)
\end{equation}

where \(k_{j}(v_{j},u_{j},\bar{w_{j}})\) is the population kernel that
describes all possible transitions from size \(u\) to \(v\) and
\(\bar{w_{j}}\) is a scalar representing the average intraspecific
crowding experienced by a genet of size \(u_j\) and species \(j\). The
integral is evaluated over all possible sizes between predefined lower
(\emph{L}) and upper (\emph{U}) size limits that extend beyond the range
of observed genet sizes.

Since the IPM is spatially-implicit, we cannot calculate neighborhood
crowding for specific genets (\(w_{ij}\)). Instead, we use an
approximation (\(\bar{w_{j}}\)) that captures the essential features of
neighborhood interactions (Adler et al. 2010). This approximation relies
on a `no-overlap' rule for conspecific genets to approximate the
overdispersion of large genets in space (Adler et al. 2010).

The population kernel is defined as the joint contributions of survival
(\emph{S}), growth (\emph{G}), and recruitment (\emph{R}):

\begin{equation}
k_{j}(v_{j},u_{j},\bar{w_{j}}) = S_j(u_j, \bar{w_{j}}(u_{j}))G_j(v_{j},u_{j},\bar{w_{j}}(u_{j})) + R_j(v_{j},u_{j},\bar{w_{j}}),
\end{equation}

which means we are calculating growth (\emph{G}) for individuals that
survive (\emph{S}) from time \emph{t} to \emph{t+1} and adding in newly
recruited (\emph{R}) individuals of an average sized one-year-old genet
for the focal species. Our stastical model for recruitment (\emph{R},
described above) returns the number of new recruit produced per quadrat.
Following previous work (Adler et al. 2012, Chu and Adler 2015), we
assume that fecundity increases linearly with size
(\(R_j(v_{j},u_{j},\bar{w_{j}}) = e^{u_j}R_j(v_{j},\bar{w_{j}})\)) to
incorporate the recruitment function in the spatially-implicit IPM.

We used random draws from the final 1,000 iterations from each of three
MCMC chains to introduce stochasticity into our population models. At
each time step, we randomly selected climate covariates from one of the
14 observed years. Then, we drew the full parameter set (climate effects
and density-dependence fixed effects) from a randomly selected MCMC
iteration. Using this approach, rather than simply using coefficient
point estimates, captures the effect of parameter uncertainty.
Relatively unimportant climate covariates (those that broadly overlap 0)
will have little effect on the mean of the simulation results, but can
contribute to their variation. Since our focus was on the contribution
of climate covariates to population states, we set the random year
effects and the random group effects to zero.

\paragraph{Quad-based model}\label{quad-based-model}

To simulate our quad-based model (QBM), we simply iterate the
quadrat-level statistical model (Eqs. 9-10). We use the same approach
for drawing parameter values as described for the IPM. After drawing the
appropriate parameter set, we calculate the mean response (population
cover at \emph{t+1} = \(x_{t+1}\)) according to Eq. 9. We then make a
random draw from a {[}0,1{]} truncated lognormal distribution with mean
equal to \(x_{t+1}\) from Eq. 9 and the variance estimate from the
fitted model. We can then project the model forward by drawing a new
parameter set (unique to climate year and MCMC iteration) at each
timestep. As with the IPM, random year effects are ignored for all
simulations.

\subsubsection{Model validation}\label{model-validation}

To test each model's ability to forecast population state, we made
out-of-sample predictions using leave-one-year-out cross validation. For
both levels of modeling, we fit the vital rate models using observations
from all years except one, and then used those fitted parameters in the
population models to perform a one-step-ahead forecast for the year
whose observations were withheld from model fitting. Within each
observation year, several quadrats were sampled. We made predictions for
each observed quadrat in the focal year, initializing each simulation
with cover in the quadrat the previous year. Since we were making
quadrat-specific predictions, we incorporated the group random effect on
the intercept for both models. We repeated this procedure for all 13
observation years, making 100 one-step-ahead forecasts for each
quadrat-year combination with parameter uncertainty included via random
draw from the MCMC chain as described above. Random year effects were
set to zero since year effects cannot be assigned to unobserved years.

This cross-validation procedure allowed us to compare accuracy and
precision of the two modeling approaches (IPM versus QBM). We first
calculated the median predicted cover across the 100 simulations for
each quadrat-year and then calculated the absolute error as the absolute
value of the difference between the observed cover for a given
quadrat-year and the median prediction. To arrive at mean absolute error
(MAE), we then averaged the absolute error within each species across
the quadrat-year specific errors. We use MAE as our measure of accuracy.
To measure precision we calculated the distance between the upper and
lower 90th quantiles of the 100 predictions and averaged this value over
quadrat-years for each species.

\subsubsection{Testing sensitivity to climate
covariates}\label{testing-sensitivity-to-climate-covariates}

With our fitted and validated models in hand, we ran simulations for
each model type (IPM and QBM) under four climate perturbation scenarios:
(1) observed climate, (2) precipitation increased by 1\%, (3)
temperature increased by 1\%, and (4) precipitation and temperature
increased by 1\%. We ran the simulations for 2,500 time steps, enough to
estimate equilibrium cover after discarding an initial 500 time steps as
burn-in. Each simulation was run under two parameter scenarios: (1)
using mean parameter estimates and (2) using randomly drawn parameters
from the MCMC chain. We use (1) to detect the overall sensitivity of
equilibrium cover to climate, and we use (2) to show the impact of model
and parameter uncertainty on forecast precision.

As an effort to identify potential discrepencies between IPM and QBM
forecasts, we also ran simulations designed to quantify the
sensitivities of individual and combined vital rates to climate for the
IPM. Specifically, we ran simulations for the above climate scenarios,
but applied the perturbed climate covariates to survival, growth, or
recruitment vital rates individually and in pairwise combinations. This
allowed us to isolate the vital rate(s) most sensitive to climate. For
this analysis, we used mean parameter estimates to reduce the sources of
uncertainty in the sensitivity estimates.

We expected the IPM to produce more accurate and precise forecasts due
to either (1) the smaller sample size of the quadrat level data sets
compared to the individual level data sets, leading to larger parameter
uncertainty for the QBM, or (2) the QBM climate effects being weakly
associated with one or more vital rate climate effects at the individual
level. To assess the impact of sample size on QBM parameter uncertainty
we refit the QBM statistical model (Eqs. 9-10) after removing sets of 2,
5, 10, and 15 quadrats. We fit 10 models at each level of quadrat
removal (2, 5, 10, 15 quadrats), removing a different randomly selected
set of quadrats for each fit. We calculated the standard deviation of
climate main effects (pptLag, ppt1, ppt2, TmeanSpr1, and TmeanSpr2) for
each model and averaged those over replicates within each set of quadrat
removals. This allowed us to regress parameter uncertainty against
sample size.

To deterime if the QBM climate effects are correlated with climate
effects for each vital rate model in the IPM, we simply regressed the
QBM climate coefficients against each vital rate model's climate
coefficients and calculated Pearson's \(\rho\). Strong correlations
indicate the QBM is capable of detecting climate effects associated with
individual vital rates. A weak correlation indicates the QBM ``misses''
the climate effect on a particular vital rate.

\subsection{Results}\label{results}

\subsubsection{Comparison of forecast
models}\label{comparison-of-forecast-models}

\subsubsection{Sensitivity of models to
climate}\label{sensitivity-of-models-to-climate}

The response of a population to climate change is a result of the
aggregate effects of climate on individual vital rates. Since the IPM
approach relies on vital rate regressions, we were able to quantify the
sensitivity of each vital rate in isolation and in pairwise
combinations. Across all species, climate covariates can have opposing
effects on different vital rates (Fig. 3). Growth was the most sensitive
vital rate for all species, showing a negative response to increased
precipitation, and stronger positive response to increased temperature,
and a mostly positive response when both climate factors are increased
(Fig. 3). \emph{B. gracilis} survival rates were sensitive to
temperature, resulting in an increase in plant cover under increased
temperature (Fig. 3a). In isolation, recruitment and survival were
insensitive to climate factors for \emph{H. comata} (Fig. 3b). Survival
and recruitment of \emph{P. smithii} were both sensitive, negatively, to
temperature and precipitation (Fig. 3c). \emph{P. secunda} equilibrium
cover was sensitive to the climate effects on survival and recruitment,
showing a negative effect on both vital rates for increased precipition,
but a strong positive effect on survival with increased temperature
(Fig. 3d). Equilibrium cover responded negatively when increased
precipitation and temperature affect recruitment (Fig. 3d). At least two
of three vital rates were sensitive to climate for each species (Fig.
3).

\subsubsection{Sources of uncertainty in the
QBM}\label{sources-of-uncertainty-in-the-qbm}

Sample size had a relatively weak effect on QBM climate parameter
uncertainty after the number of quadrats used in fitting exceeded about
10 (Fig. 5). Inverse-gaussian fits show that increasing sample size
beyond the number of quadrats we used would result in diminishing
returns in terms of parameter certainty (Fig. 5).

Climate effects estimated from the QBM are most correlated with climate
effects from the growth regression at the individual level (Fig. 6). In
no case does the QBM statistical model have strong correlations across
all three vital rates (Fig. 6). QBM climate effects were most weakly
correlated with those from individual-level recruitment models for
\emph{B. gracilis}, \emph{H. comata}, and \emph{P. secunda} (Fig.
6a,b,d). For \emph{P. smithii}, QBM climate effects showed no
correlation with the survival model effects (Fig. 6c).

\subsubsection{Model forecasts}\label{model-forecasts}

Forecasts based on 1\% climate changes were extremely uncertain when we
considered model error and parameter uncertainty (Fig. 6; simulations
with mean parameters are in Appendix D for comparison). As expected
based on model validation (Table 1), QBM projections were more uncertain
than IPM projections for all species except \emph{P. smithiii} (Fig. 6).
IPM forecasts for \emph{P. smithiii} were very uncertain due to a very
high instrinsic rate of recruitment combined with uncertainty in climate
coefficients which lead to high recruitment boom years and subsequent
busts when young plants suffer high mortality (Appendx C). When we
included model error and parameter uncertainty, forecast changes in
proportional cover always spanned a wide range of negative to positive
values. In other words, neither model could predict whether a climate
perturbation would increase or decrease equilibrium population size.

\subsection{Discussion}\label{discussion}

Population models built using individual-level data allow inference on
demographic processes, but they can only forecast future population
states across the (typically limited) spatial extent of the
observations. Population-level data are much easier to collect across
broad spatial extents, so models built using such data offer an
appealing alternative to traditional population models (Queenborough et
al. 2011). However, density-structured models rely on the aggregation of
individual-level data. This creates a potential problem if such models
are to be used in a climate change context because it is individuals,
not populations, which respond to climate (Clark et al. 2012). Are
models based on population-level metrics as sensitive to climate as
models based on individual-level metrics? Do these two types of models
produce consistent forecasts? Do we need detailed demographic data to
forecast the impacts of climate change?

\subsubsection{The importance of demographic
data}\label{the-importance-of-demographic-data}

Our comparison of a traditional, demographic population model (the IPM)
with a model inspired by density-structured models (the QBM) showed that
the IPM outperformed the QBM: the IPM was more accurate and precise than
the QBM in out-of-sample cross validation (Table 1). The superiority of
the IPM could reflect either differences in sample size or the effect of
averaging over unique effects of climate on each individual-level vital
rate. Although increasing sample size of quadrat percent-cover
observations would be easy to do in the field, we found little evidence
that it would lead to higher precision of climate coefficient estimates
(Fig. 4).

We did, however, find evidence that the QBM statistical model failed to
identify climate dependence for some vital rates (Fig. 5). For no
species were climate effects from the QBM strongly correlated with all
three vital rates (Fig. 5). Freckleton et al. (2011) acknowledge that
averaging over complex stage dependence will lead to poorly specified
models. This is analagous to our situation, but instead of averaging
over complex life histories, we are averaging over complex climate
dependence. Though our work here focused on plant species, this finding
is applicable to any species with vital rates that respond uniquely to
weather/climate.

Our interpretation is that the QBM is ``missing'' climate signals
associated with at least one vital rate for each species. This leads to
inaccurate and imprecise forecasts because the QBM statistical model
struggles to explain variation due to climate variables that have
positive and negative impacts on different vital rates. When this is the
case, as it is for all our species to varying degrees (Fig. 3),
forecasts from models based on population-level data will fail. Our
result is consistent with related work on the importance of
individual-level data to forecast population responses to exogenous
drivers (Clark et al. 2011a, 2011b, 2012, Galv{á}n et al. 2014).

Detailed demographic data appears to be necessary to forecast climate
change impacts on plant populations when vital rates have unique climate
responses. How then can we build models to make forecasts for the
landscape and regional scales beyond the scope of traditional population
models (Queenborough et al. 2011)? There are alternatives to
density-structured models. For example, Clark et al. (2011a) use Forest
Inventory and Analysis (FIA) data to parameterize a population model
with multiple vital rates and climate dependence. Distributed efforts
such as PlantPopNet (\url{http://plantago.plantpopnet.com}) will allow
researchers to estimate variation around climate responses for
widespread species by taking advantage of spatial variation in climate
(e.g. Doak and Morris 2010). Finally, new approaches on the horizon that
leverage photo/video of plots and advanced object recognition algorithms
(e.g. Liu et al. 2014) will increase the efficiency of plant mapping and
digitizing efforts.

\subsubsection{The challenge of
uncertainty}\label{the-challenge-of-uncertainty}

An important, but unexpected, result of our analysis was the great
uncertainty in forecasts, even for our best model. The typical approach
in ecology is to use point estimates of model parameters to project
populations forward according to the specified model, usually allowing
for some variability around the determinstic process (e.g. Battin et al.
2007, Jenouvrier et al. 2009, Adler et al. 2012). If we follow tradition
and calculate the mean response to climate perturbation with only model
error and interannual variation included, the IPM and the QBM produce
opposing forecasts for three of four species (Fig. D1). It would be
tempting to interpret this inconsistency as further evidence for the
superiority of the IPM. However, if we introduce parameter uncertainty,
the forecasts are actually indistinguishable (Fig. 6), though the IPM
projections are generally more precise (consistent with our
cross-validation results). The real story is that both models produce
highly uncertain forecasts. For all species, the 90\% quantiles of
predicted changes in population size overlapped zero; we cannot even
predict whether a change in precipitation or temperature will cause
populations to increase or decrease. This result held when we tried
perturbing climate by 10\% and 20\% as well.

Our results highlight the state of affairs in ecology when it comes to
forecasting the impacts of climate change. The analysis we conducted
here could be considered at the forefront of ecological forecasting with
respect to the statistical approach employed (hierarchical Bayesian),
the type of population model we used (density-dependent, stochastic IPM
with parameter uncertainty), and the amount of high quality data we had
at our disposal (14 years of individual-level data). Yet, model
predictions proved so uncertain that any forecast, when bounded with
model and parameter uncertainty, would be uninformative.

How might we improve on this state of affairs? First, forecasts could be
improved by matching the spatial scale of predictor variables with the
spatial scale of observations. One of the major limitations of the
models we fit here is that the climate data are collected at a larger
scale than the individual-level observations of plant size. Climate
covariates only vary by year, with no spatial variability within years.
Thus, even if we fit models to individual-level data, we are missing the
key interaction point between weather and individual plants (Clark et
al. 2011b) because all observations share the same climate covariates.
Demographic studies should be designed with at least plot-level
measurements of climate related variables (e.g., soil moisture). Second,
accurately detecting climate signals will take even longer time series.
Recent theoretical work on detecting climate signals in noisy data
suggests that even advanced approaches to parameter fitting require
20-25 year time series (Teller et al. n.d.). Third, ecologists need a
stronger commitment to reporting uncertainty. Although most modeling
studies explicitly consider model uncertainty, parameter uncertainty is
often ignored. In some cases this is because the most convenient
statistical methods make it difficult to propogate parameter
uncertainty. Yet even Bayesian approaches that allow integration of
model fitting and forecasting (Hobbs and Hooten 2015) are not simple
when using modeling approaches like integral projection models that
separate the model fitting and simulation stages (Rees and Ellner 2009).
However, as we have done here, it is still possible to include parameter
uncertainty by drawing parameter values from MCMC iterations, taking
care to draw all parameters from the same chain and iteration to account
for their correlations. Only by being honest about our forecasts can we
begin to produce better ones, and forecasts reported without parameter
error are disingenuous. Ignoring parameter error may be justifiable when
the goal is investigating basic processes, but it is indefensible when
forecasting is the goal.

\subsection{Conclusions}\label{conclusions}

This work is not a critique of density-structured population models. We
are confident that density-structured models will prove to be a valuable
tool for many applications. However, our analysis represents the first
comparison, to our knowledge, of population models based on individual
and aggregated forms of the same data in a climate change context. Our
results confirm theoretical arguments (Clark et al. 2011b) and empirical
evidence (Clark et al. 2011a, 2012) that individual responses are
critical for predicting species' responses to climate change. It seems
there is no short cut to producing accurate and precise population
forecasts: we need detailed demographic data to forecast the impacts of
climate change on populations. Given the importance of demographic data
and its current collection cost, we need modern methods to collect
demographic data more efficiently across environmental gradients in
space and time.

Our results also offer a cautionary tale because forecast uncertainty
was large for both model types. Even with 14 years of detailed
demographic data and sophisticated modeling techniques, our projections
contained too much uncertainty to be informative. Uncertainty in
demographic responses to climate can be reduced by collecting (1) longer
time series and (2) climate covariates that match the scale of inference
(e.g., plot rather than landscape level climate/weather metrics).

\subsection{Acknowledgments}\label{acknowledgments}

This work was funded by the National Science Foundation through a
Postdoctoral Research Fellowship in Biology to ATT (DBI-1400370) and a
CAREER award to PBA (DEB-1054040). We thank the original mappers of the
permanent quadrats in Montana and the digitizers in the Adler lab,
without whom this work would not have been possible. Informal
conversations with Stephen Ellner, Giles Hooker, Robin Snyder, and a
series of meetings between the Adler and Weecology labs at USU sharpened
our thinking. Brittany Teller provided comments that improved our
manuscript. Compute, storage and other resources from the Division of
Research Computing in the Office of Research and Graduate Studies at
Utah State University are gratefully acknowledged.

\pagebreak{}

\setcounter{page}{29}

\subsection{Tables}\label{tables}

\pagebreak{}

\subsection{Figure Legends}\label{figure-legends}

Figure 1. Time series of average percent cover over all quadrats for our
four focal species: \emph{Bouteloua gracilis} (BOGR), \emph{Hesperostipa
comata} (HECO), \emph{Pascopyrum smithii} (PASM), and \emph{Poa secunda}
(POSE). Light grey lines show trajectories of individual quadrats. Note
the different y-axis scales across panels.

Figure 2. Work flow of the data aggregation, model fitting, and
population simulating.

Figure 3. Sensitivity of equilibrium cover simulated from the IPM to
each climate scenario applied to individual and combined vital rates.
For example, the points associated with G show the median cover from IPM
simulations where a climate perturbation is applied only to the growth
regression climate covariates. These simulations use mean parameter
values for clarity.

Figure 4. Effect of quadrat sample size on the precision (standard
deviation) of main climate effect estimates in the QBM. Increasing the
number of quadrats results in diminishing returns in terms of parameter
certainty. Light dashed lines show individual climate effects at five
quadrat sample sizes. Thick dark lines are inverse gaussian fits showing
the mean effect of increasing quadrat sample size on parameter
precision.

Figure 5. Correlations (r) between QBM and IPM estimates of climate
effects. We ignore sizeXclimate interactions since these are not
directly comparable across model types. The QBM does not have multiple
vital rates, so its values are repeated across panels within each
species. Across top panels, `growth' = growth regression, `rec' =
recruitment regression, `surv' = survival regression.

Figure 6. Mean (points) and 90\% quantiles (errorbars) for the
proportional difference between baseline simulations (using observed
climate) and the climate pertubation simulation on the x-axis. We
calculated proportional difference as log(perturbed climate cover) -
log(observed climate cover), where `perturbed' and `observed' refer to
the climate time series used to drive interannual variation in the
simulations. Model error and parameter uncertainty were propagated
through the simulation phase. Climate simulations are as in Figure 3.

\pagebreak{}

\subsection{Figures}\label{figures}

\pagebreak{}

\setcounter{page}{24}

\subsection*{References}\label{references}
\addcontentsline{toc}{subsection}{References}

Adler, P. B., H. J. Dalgleish, and S. P. Ellner. 2012. Forecasting plant
community impacts of climate variability and change: when do competitive
interactions matter? Journal of Ecology 100:478--487.

Adler, P. B., S. P. Ellner, and J. M. Levine. 2010. Coexistence of
perennial plants: An embarrassment of niches.

Anderson, J., L. Vermeire, and P. B. Adler. 2011. Fourteen years of
mapped, permanent quadrats in a northern mixed prairie, USA. Ecology
92:1703.

Ara{ú}jo, M. B., and A. T. Peterson. 2012. Uses and misuses of
bioclimatic envelope modeling. Ecology 93:1527--1539.

Battin, J., M. W. Wiley, M. H. Ruckelshaus, R. N. Palmer, E. Korb, K. K.
Bartz, and H. Imaki. 2007. Projected impacts of climate change on salmon
habitat restoration. Proceedings of the National Academy of Sciences of
the United States of America 104:6720--6725.

Chu, C., and P. B. Adler. 2014. When should plant population models
include age structure? Journal of Ecology 102:531--543.

Chu, C., and P. B. Adler. 2015. Large niche differences emerge at the
recruitment stage to stabilize grassland coexistence. Ecological
Monographs 85:373--392.

Chu, C., K. M. Havstad, N. Kaplan, W. K. Lauenroth, M. P. McClaran, D.
P. Peters, L. T. Vermeire, and P. B. Adler. 2014. Life form influences
survivorship patterns for 109 herbaceous perennials from six semi-arid
ecosystems. Journal of Vegetation Science 25:947--954.

Clark, J. S., and O. N. Bj{ø}rnstad. 2004. Population time series:
Process variability, observation errors, missing values, lags, and
hidden states. Ecology 85:3140--3150.

Clark, J. S., D. M. Bell, M. H. Hersh, and L. Nichols. 2011a. Climate
change vulnerability of forest biodiversity: Climate and competition
tracking of demographic rates. Global Change Biology 17:1834--1849.

Clark, J. S., D. M. Bell, M. H. Hersh, M. C. Kwit, E. Moran, C. Salk, A.
Stine, D. Valle, and K. Zhu. 2011b. Individual-scale variation,
species-scale differences: Inference needed to understand diversity.

Clark, J. S., D. M. Bell, M. Kwit, A. Stine, B. Vierra, and K. Zhu.
2012. Individual-scale inference to anticipate climate-change
vulnerability of biodiversity. Philosophical Transactions of the Royal
Society B: Biological Sciences 367:236--246.

Clark, J. S., D. Bell, C. Chu, B. Courbaud, M. Dietze, M. Hersh, J.
HilleRisLambers, I. Ib{á}{ñ}ez, S. LaDeau, S. McMahon, J. Metcalf, J.
Mohan, E. Moran, L. Pangle, S. Pearson, C. Salk, Z. Shen, D. Valle, and
P. Wyckoff. 2010. High-dimensional coexistence based on individual
variation: a synthesis of evidence. Ecological Monographs 80:569--608.

Clark, J. S., S. R. Carpenter, M. Barber, S. Collins, A. Dobson, J. A.
Foley, D. M. Lodge, M. Pascual, R. Pielke, W. Pizer, C. Pringle, W. V.
Reid, K. A. Rose, O. Sala, W. H. Schlesinger, D. H. Wall, and D. Wear.
2001. Ecological forecasts: an emerging imperative. Science (New York,
N.Y.) 293:657--660.

Clark, J. S., A. E. Gelfand, C. W. Woodall, and K. Zhu. 2014. More than
the sum of the parts: Forest climate response from joint species
distribution models. Ecological Applications 24:990--999.

Dalgleish, H. J., D. N. Koons, M. B. Hooten, C. A. Moffet, and P. B.
Adler. 2011. Climate influences the demography of three dominant
sagebrush steppe plants. Ecology 92:75--85.

Doak, D. F., and W. F. Morris. 2010. Demographic compensation and
tipping points in climate-induced range shifts. Nature 467:959--962.

Elith, J., and J. R. Leathwick. 2009. Species Distribution Models:
Ecological Explanation and Prediction Across Space and Time.

Ellner, S. P., and M. Rees. 2006. Integral projection models for species
with complex demography. The American naturalist 167:410--428.

Freckleton, R. P., W. J. Sutherland, A. R. Watkinson, and S. A.
Queenborough. 2011. Density-structured models for plant population
dynamics. American Naturalist 177:1--17.

Galv{á}n, J. D., J. J. Camarero, and E. Guti{é}rrez. 2014. Seeing the
trees for the forest: Drivers of individual growth responses to climate
in Pinus uncinata mountain forests. Journal of Ecology 102:1244--1257.

Guisan, A., and W. Thuiller. 2005. Predicting species distribution:
Offering more than simple habitat models.

Hobbs, N. T., and M. B. Hooten. 2015. Bayesian Models: A Statistical
Primer for Ecologists. Princeton University Press, Princeton.

Jenouvrier, S., H. Caswell, C. Barbraud, M. Holland, J. Stroeve, and H.
Weimerskirch. 2009. Demographic models and IPCC climate projections
predict the decline of an emperor penguin population. Proceedings of the
National Academy of Sciences of the United States of America
106:1844--1847.

Lauenroth, W. K., and P. B. Adler. 2008. Demography of perennial
grassland plants: Survival, life expectancy and life span. Journal of
Ecology 96:1023--1032.

Liu, Y., Y. Jang, W. Woo, and T.-K. Kim. 2014. Video-Based Object
Recognition Using Novel Set-of-Sets Representations.

Luo, Y., K. Ogle, C. Tucker, S. Fei, C. Gao, S. LaDeau, J. S. Clark, and
D. S. Schimel. 2011. Ecological forecasting and data assimilation in a
data-rich era. Ecological Applications 21:1429--1442.

Maiorano, L., R. Cheddadi, N. E. Zimmermann, L. Pellissier, B.
Petitpierre, J. Pottier, H. Laborde, B. I. Hurdu, P. B. Pearman, A.
Psomas, J. S. Singarayer, O. Broennimann, P. Vittoz, A. Dubuis, M. E.
Edwards, H. A. Binney, and A. Guisan. 2013. Building the niche through
time: using 13,000 years of data to predict the effects of climate
change on three tree species in Europe. Global Ecology and Biogeography
22:302--317.

Merow, C., A. M. Latimer, A. M. Wilson, S. M. McMahon, A. G. Rebelo, and
J. A. Silander. 2014. On using integral projection models to generate
demographically driven predictions of species' distributions:
development and validation using sparse data. Ecography 37:1167--1183.

Pearson, R. G., and T. P. Dawson. 2003. Predicting the impacts of
climate change on the distribution of species: Are bioclimate envelope
models useful? Global Ecology and Biogeography 12:361--371.

Perretti, C. T., S. B. Munch, and G. Sugihara. 2013. Model-free
forecasting outperforms the correct mechanistic model for simulated and
experimental data. Proceedings of the National Academy of Sciences of
the United States of America 110:5253--5257.

Petchey, O. L., M. Pontarp, T. M. Massie, S. K{é}fi, A. Ozgul, M.
Weilenmann, G. M. Palamara, F. Altermatt, B. Matthews, J. M. Levine, D.
Z. Childs, B. J. McGill, M. E. Schaepman, B. Schmid, P. Spaak, A. P.
Beckerman, F. Pennekamp, and I. S. Pearse. 2015. The ecological forecast
horizon, and examples of its uses and determinants. Ecology Letters
18:597--611.

Piantadosi, S., D. P. Byar, and S. B. Green. 1988. The Ecological
Fallacy. American Journal of Epidemiology 127:893--904.

Queenborough, S. A., K. M. Burnet, W. J. Sutherland, A. R. Watkinson,
and R. P. Freckleton. 2011. From meso- to macroscale population
dynamics: A new density-structured approach. Methods in Ecology and
Evolution 2:289--302.

R Core Development Team. 2013. R: A language and environment for
statistical computing.

Rees, M., and S. P. Ellner. 2009. Integral projection models for
populations in temporally varying environments. Ecological Monographs
79:575--594.

Schurr, F. M., J. Pagel, J. S. Cabral, J. Groeneveld, O. Bykova, R. B.
O'Hara, F. Hartig, W. D. Kissling, H. P. Linder, G. F. Midgley, B.
Schr{ö}der, A. Singer, and N. E. Zimmermann. 2012. How to understand
species' niches and range dynamics: A demographic research agenda for
biogeography. Journal of Biogeography 39:2146--2162.

Stan Development Team. 2014a. Stan: A C++ Library for Probability and
Sampling, Version 2.5.0.

Stan Development Team. 2014b. Rstan: the R interface to Stan, Version
2.5.0.

S{æ}ther, B. E., S. Engen, V. Gr{ø}tan, W. Fiedler, E. Matthysen, M. E.
Visser, J. Wright, A. P. M{ø}ller, F. Adriaensen, H. {Van Balen}, D.
Balmer, M. C. Mainwaring, R. H. McCleery, M. Pampus, and W. Winkel.
2007. The extended Moran effect and large-scale synchronous fluctuations
in the size of great tit and blue tit populations. Journal of Animal
Ecology 76:315--325.

Taylor, C. M., and A. Hastings. 2004. Finding optimal control strategies
for invasive species: a density-structured model for Spartina
alterniflora. Journal of Applied Ecology 41:1049--1057.

Teller, B. J., P. B. Adler, C. B. Edwards, G. Hooker, R. E. Snyder, and
S. P. Ellner. (n.d.). Linking demography with drivers: climate and
competition. Methods in Ecology and Evolution.

\end{document}
